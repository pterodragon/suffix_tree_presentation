\documentclass[compress,usenames,dvipsnames]{beamer}
\usepackage[T1]{fontenc}
\usepackage{beramono}
\usepackage{xcolor}
\usepackage[ruled]{algorithm2e}
\usepackage{tikz}
\usetikzlibrary{shapes,snakes}
\usetikzlibrary{arrows.meta}
\usepackage{hyperref}
\usetheme{Warsaw}
\usecolortheme{crane}

\SetKw{Not}{not}
\SetKw{True}{true}
\SetKw{False}{false}
\mathchardef\mhyphen="2D

\newcommand*\diamonded[1]{\tikz[baseline=(char.base)]{
\node[shape=diamond,draw,inner sep=2pt] (char) {#1};}}

\newcommand*\circled[1]{\tikz[baseline=(char.base)]{
\node[shape=circle,draw,inner sep=2pt] (char) {#1};}}

\let\oldnl\nl% Store \nl in \oldnl
\newcommand{\nonl}{\renewcommand{\nl}{\let\nl\oldnl}}% Remove line number for one line

\DeclareMathOperator{\atrans}{\mathnormal{-}}

\def\HiLi{\leavevmode\rlap{\hbox to \hsize{\color{yellow!50}\leaders\hrule height .8\baselineskip depth .5ex\hfill}}}
\usefonttheme[onlymath]{serif}

\newenvironment{definitionblock}[1]{
    \setbeamercolor{block title}{bg=cyan}
    \begin{block}{#1}}{\end{block}
}

\newenvironment{theoremblock}[1]{
    \setbeamercolor{block title}{bg=Emerald}
    \begin{block}{#1}}{\end{block}
}

\newenvironment{factblock}[1]{
    \setbeamercolor{block title}{bg=Apricot}
    \begin{block}{#1}}{\end{block}
}

\author{Wing}
\title{Suffix Tree (Ukkonen's algorithm)}  

\begin{document}
\date{\today} 

\frame[plain]{\titlepage} % [plain] means it doesn't show the section above the Header 

\frame[plain]{\frametitle{Table of contents}
    \small
    \tableofcontents[hideallsubsections]
}  

\section{Suffix Trie}
\begin{frame}[fragile,plain] \frametitle{Trie}
    \begin{block}{Trie}
        An ordered tree data structure used to store a dynamic set or map where the keys are usually strings
    \end{block}
\end{frame}

\begin{frame}[fragile,plain] \frametitle{Suffix Trie}
    \begin{figure}
        \begin{tikzpicture}[level distance=10mm,->,line width=1.5pt]
            \tikzstyle{every node}=[draw,circle,inner sep=1pt,minimum size=7.5pt,line width=0.5pt]
            \tikzstyle{sll}=[bend left,dashed,line width=0.5pt,->]
            \tikzstyle{slr}=[bend right,dashed,line width=0.5pt,->]
            \tikzstyle{level 1}=[sibling distance=10mm]
            \tikzstyle{level 2}=[sibling distance=20mm]
            \tikzstyle{level 3}=[sibling distance=10mm]
            \node [draw=none]
                (bottom) {$\bot$}
                child {node (root) {}
                    child {node {}
                        child{node {}
                            child{node {}
                                child{node {}
                                    child{node {}
                                        edge from parent
                                        node[auto,draw=none] {$o$}
                                    }
                                    child{node[draw=none] {}
                                        edge from parent[draw=none]
                                        node[auto,draw=none] {}
                                    }
                                    edge from parent
                                    node[auto,draw=none] {$a$}
                                }
                                child{node[draw=none] {}
                                    edge from parent[draw=none]
                                    node[auto,draw=none] {}
                                }
                                edge from parent
                                node[auto,draw=none] {$c$}
                            }
                            child{node {}
                                edge from parent
                                node[auto,draw=none] {$o$}
                            }
                            edge from parent
                            node[auto,draw=none] {$a$}
                        }
                        child{node[draw=none] {}
                            edge from parent[draw=none]
                            node[auto,draw=none] {}
                        }
                        edge from parent
                        node[auto,draw=none] {$c$}
                    }
                    child {node {}
                        child{node {}
                            child{node {}
                                child{node {}
                                    edge from parent
                                    node[auto,draw=none] {$o$}
                                }
                                child{node[draw=none] {}
                                    edge from parent[draw=none]
                                    node[auto,draw=none] {}
                                }
                                edge from parent
                                node[auto,draw=none] {$a$}
                            }
                            child{node[draw=none] {}
                                edge from parent[draw=none]
                                node[auto,draw=none] {}
                            }
                            edge from parent
                            node[auto,draw=none] {$c$}
                        }
                        child{node {}
                            edge from parent
                            node[auto,draw=none] {$o$}
                        }
                        edge from parent
                        node[auto,draw=none] {$a$}
                    }
                    child {node {}
                        edge from parent
                        node[auto,draw=none] {$o$}
                    }
                    edge from parent
                    node[draw=none,left] {$\Sigma$}
                }
                ;
                % \draw[style=slr] (root) to (bottom);
            \end{tikzpicture}
            \caption{Suffix Trie for $``cacao"$ (Suffix links omitted)} \label{fig:M1}
        \end{figure}
    \end{frame}

    \begin{frame}[fragile,plain] \frametitle{Suffix Trie}
        \begin{itemize}
            \item Proposed by Esko Ukkonen (University of Helsinki, Finland)
            \item An algorithm easier to grasp than the those in the literature at that time
            \item On-line algorithm: Processes the string symbol by symbol from left to right, and always has the suffix tree for the scanned
                part of the string ready
        \end{itemize}
    \end{frame}

    \begin{frame}[fragile,plain] \frametitle{Construction of Suffix Trie}
        \begin{definitionblock}{String}
            Let $T = t_1t_2\cdots t_n$ be a string over alphabet $\Sigma$
        \end{definitionblock}
        \begin{definitionblock}{Substring}
            Each string $x : T = uxv$ for some (possibly empty) string $u$ and $v$ is a \underline{substring} of $T$
        \end{definitionblock}
        \begin{definitionblock}{Suffix}
            $T_i = t_i\cdots t_n $ where $ 1 \leq i \leq n + 1 $
            \begin{itemize}
                \item $T_{n+1} = \epsilon$ is the \emph{empty} suffix
            \end{itemize}
        \end{definitionblock}
    \end{frame}

    \begin{frame}[fragile,plain] \frametitle{Construction of Suffix Trie}
        \begin{definitionblock}{Set of all suffixes of $T$}
            $\sigma(T)$
        \end{definitionblock}
        The suffix trie of $T$ is a trie representing $\sigma(T)$
    \end{frame}

    \begin{frame}[fragile,plain] \frametitle{Construction of Suffix Trie}
        \begin{definitionblock}{Suffix Trie}
            Denote suffix trie of $T$ as $STrie(T) = (Q \cup \{\bot\}, root, F, g, f)$
            \hfill \break
            \hfill \break
            Define such a trie as an augmented deterministic finite-state automaton which has a tree-shaped transition graph representing the trie for $\sigma(T)$
            \hfill \break
            \hfill \break
            augmented with
            \begin{itemize}
                \item $f$ : \underline{suffix function}
                \item $\bot$ : \underline{auxiliary state}
            \end{itemize}
        \end{definitionblock}
    \end{frame}

    \begin{frame}[fragile,plain] \frametitle{Construction of Suffix Trie}
        \begin{definitionblock}{Set $Q$ of the states of $STrie(T)$}
            The set $Q$ of the states of $STrie(T)$ can be put in a one-to-one correspondence with the substrings of $T$.
            \hfill \break
            \hfill \break
            Denote $\bar{x}$ the state that corresponds to a substring $x$
            \hfill \break
            Shorthand: $\bar{x} \sim x$
            \begin{itemize}
                \item $root \sim \epsilon$
                \item set $F$ of final states $\sim \sigma(T) $
            \end{itemize}

        \end{definitionblock}
    \end{frame}

    \begin{frame}[fragile,plain] \frametitle{Construction of Suffix Trie}
        \begin{definitionblock}{Transition function $g$}
            $
            \begin{cases}
                g(\bar{x}, a) = \bar{y} \  & \forall \bar{x}, \bar{y} \in Q : y = xa, \ \mbox{where} \ a \in \Sigma
                \\
                g(\bot, a) = root \ & \forall a \in \Sigma
            \end{cases}
            $
        \end{definitionblock}
        \begin{definitionblock}{Suffix function $f$}
            $\forall \bar{x} \in Q,$ \\
            $
            \begin{cases}
                f(\bar{x}) = \bar{y} & \mbox{if} \ \bar{x} \neq root, \mbox{then}\ x = ay, a \in \Sigma \\
                f(root) = \bot \\
                f(\bot) \ \mbox{is undefined}
            \end{cases}
            $
        \end{definitionblock}
        $\bot \sim a^{-1} \ \forall a \in \Sigma$
        \hfill \break
        $a^{-1}a=\epsilon$
    \end{frame}

    \begin{frame}[fragile,plain] \frametitle{Construction of Suffix Trie}
        \begin{definitionblock}{Suffix Link}
            $f(r)$ is the suffix link of state $r$
        \end{definitionblock}

        \begin{definitionblock}{Prefix}
            $T^{i} = t_1\cdots t_i$ of $T$ for $0 \leq i \leq n $
        \end{definitionblock}
    \end{frame}

    \begin{frame}[fragile,plain] \frametitle{Construction of Suffix Trie}
        \begin{block}{Key observation}
            How is $STrie(T^i)$ obtained from $STrie(T^{i-1})$?
            \hfill \break
            \hfill \break
            The suffixes of $T^i$ can be obtained by catenating $t_i$ to the end of each suffix of $T^{i-1}$ and by adding an empty suffix, i.e.
            $$\sigma(T^i) = \sigma(T^{i-1})t_i \cup \{ \epsilon \} $$
        \end{block}
        $STrie(T^{i-1})$ accepts $\sigma(T^{i-1})$, to make it accept $\sigma(T^{i})$,
        examine $F_{i-1}$ of $STrie(T^{i-1})$
        \begin{itemize}
            \item $r \in F_{i-1}$ doesn't have a $t_i$-transition $\Rightarrow$ add transition $r \rightarrow$ new state
            \item $r \in F_{i-1}$ has a $t_i$-transition $\Rightarrow$ follow the transition to the old state
            \item All such states plus $root$ will be $F_i$ of $STrie(T^i)$
        \end{itemize}
    \end{frame}

    \begin{frame}[fragile,plain] \frametitle{Construction of Suffix Trie}
        How to find states $r \in F_{i-1}$ that get new transitions?
        \hfill \break
        \hfill \break
        From definition of the suffix function $f$,
        \hfill \break
        $r \in F_{i-1} \Leftrightarrow r = f^j (\overline{t_1\cdots t_{i-1}})$ for some $ 0 \leq j \leq i - 1$
        \begin{definitionblock}{Boundary path}
            Boundary path of $STrie(T^{i-1})$: \\
            Path starting from deepest state $\overline{t_1\cdots t_{i-1}}$ of $STrie(T^{i-1})$, following the suffix links and ending at $\bot$
        \end{definitionblock}
        $\therefore $ All states in $F_{i-1}$ are on the \underline{boundary path} of $STrie(T^{i-1})$
    \end{frame}

    \begin{frame}[fragile,plain] \frametitle{Construction of Suffix Trie}
        The boundary path is traversed. \\
        \hfill \break
        If a state $\bar{z}$ on the boundary path does not have a transition on $t_i$ yet,
        add a new state $\overline{zt_i}$ and a new transition $g(\bar{z}, t_i) = \overline{zt_i}$ \\
        \hfill \break
        To update $f$, new states $\overline{zt_i}$ are linked together with new suffix links starting from $\overline{t_1\cdots t_{i}}$. \\
        Obviously, this is the boundary path of $STrie(T^i)$
    \end{frame}

    \begin{frame}[fragile,plain] \frametitle{Construction of Suffix Trie}
        \begin{block}{Observation}
            The traversal over $F_{i-1}$ along the boundary path can be stopped immediately when the first state $\bar{z}$ is found s.t. state $\overline{zt_i}$ (and hence also transition $g(\bar{z}, t_i) = \overline{zt_i}$) already exists.
        \end{block}
        Let namely $\overline{zt_i}$ already be a state. \\
        \hfill \break
        Then $STrie(T^{i-1})$ has to contain state $\overline{z't_i}$ and transition $g(z', t_i) = \overline{z't_i} \ \forall z' = f^j(\bar{z}), j \leq 1$. \\
        In other words, if $\overline{zt_i}$ is a substring of $T_{i-1}$ then every suffix of $\overline{zt_i}$ is a substring of $T_{i-1}$.\\
        \hfill \break
        Such $\bar{z}$ must exist as $\bot$ is the last state on the boundary path that has the $t_i$-transition $\forall t_i$
    \end{frame}

    \begin{frame}[fragile,plain]\frametitle{Construction of Suffix Trie}
        $top$ denotes the state $\overline{t_1\cdots t_{i-1}}$
        \LinesNumbered
        \begin{algorithm}[H]
            \SetAlgoNoEnd
            $r \leftarrow top$\;
            \While{$g(r, t_i)$ is undefined}{
                create new state $r'$ and new transition $g(r, t_i) = r'$\;
                \lIf{$r \neq top$}{create new suffix link $f(oldr') = r'$}
                $oldr' \leftarrow r'$\;
                $r \leftarrow f(r)$\;
            }
            create new suffix link $f(oldr') = g(r, t_i)$\;
            $top \leftarrow g(top, t_i)$.
            \caption{}
            \label{alg:1}
        \end{algorithm}
    \end{frame}

    \begin{frame}[fragile,plain]\frametitle{Construction of Suffix Trie}
        Running Algorithm~\ref{alg:1} for $t_i = t_1, t_2,\cdots, t_n$ visits each $\bar{x} \in Q$ once.
        \begin{theoremblock}{Theorem 1}
            Suffix trie $STrie(T)$ can be constructed in time proportional to
            the size of $STrie(T)$ which, in the worst case, is $\mathcal{O}(|T|^2)$.
        \end{theoremblock}
    \end{frame}

    \section{Suffix Tree}
    \begin{frame}[fragile,plain]\frametitle{Suffix Tree}
        \begin{definitionblock}{Suffix tree}
            Suffix tree $STree(T)$ of $T$ is a data structure that represents $STrie(T)$ in space linear in the length $|T|$ of $T$
        \end{definitionblock}
    \end{frame}

    
\begin{frame}[fragile,plain] \frametitle{Suffix Tree}
    \begin{figure}
        \begin{tikzpicture}[remember picture,overlay,level distance=50pt,->,line width=1.5pt]
            \tikzstyle{every node}=[draw,circle,inner sep=1pt,minimum size=7.5pt,line width=0.5pt]
            \tikzstyle{sll}=[bend left,dashed,line width=0.5pt,-{>[length=10pt]}]
            \tikzstyle{slr}=[bend right,dashed,line width=0.5pt,->]
            \tikzstyle{level 1}=[sibling distance=100pt]
            \tikzstyle{level 2}=[level distance=50pt,sibling distance=120pt]
            \tikzstyle{level 3}=[sibling distance=50pt]
\node[yshift=-30pt](bottom)at (current page.north) {  } child { node(q) {  } edge from parent node[rectangle,draw=green,fill=white] { $\Sigma$ }   } ;\draw[style=sll](q) to (bottom);
\node[rectangle,draw=none,yshift=10pt] at (current page.south) {
                    \parbox{.5\textwidth}{
                    \begin{figure}
                        \caption{Suffix Tree for ``$cacao$''}
                    \end{figure}
                    }
            };
        \end{tikzpicture}
    \end{figure}
\end{frame}

    
\begin{frame}[fragile,plain] \frametitle{Suffix Tree}
    \begin{figure}
        \begin{tikzpicture}[remember picture,overlay,level distance=50pt,->,line width=1.5pt]
            \tikzstyle{every node}=[draw,circle,inner sep=1pt,minimum size=7.5pt,line width=0.5pt]
            \tikzstyle{sll}=[bend left,dashed,line width=0.5pt,-{>[length=10pt]}]
            \tikzstyle{slr}=[bend right,dashed,line width=0.5pt,->]
            \tikzstyle{level 1}=[sibling distance=100pt]
            \tikzstyle{level 2}=[level distance=50pt,sibling distance=120pt]
            \tikzstyle{level 3}=[sibling distance=50pt]
\node[yshift=-30pt](bottom)at (current page.north) {  } child { node(q) {  } child { node(q0cw) {  } edge from parent node[rectangle,draw=green,fill=white,align=center] { $cacao$ \\ $(1,5)$ }   }edge from parent node[rectangle,draw=green,fill=white] { $\Sigma$ }   } ;\draw[style=sll](q) to (bottom);
\node[rectangle,draw=none,yshift=10pt] at (current page.south) {
                    \parbox{.5\textwidth}{
                    \begin{figure}
                        \caption{Suffix Tree for ``$cacao$''}
                    \end{figure}
                    }
            };
        \end{tikzpicture}
    \end{figure}
\end{frame}

    
\begin{frame}[fragile,plain] \frametitle{Suffix Tree}
    \begin{figure}
        \begin{tikzpicture}[remember picture,overlay,level distance=50pt,->,line width=1.5pt]
            \tikzstyle{every node}=[draw,circle,inner sep=1pt,minimum size=7.5pt,line width=0.5pt]
            \tikzstyle{sll}=[bend left,dashed,line width=0.5pt,-{>[length=10pt]}]
            \tikzstyle{slr}=[bend right,dashed,line width=0.5pt,->]
            \tikzstyle{level 1}=[sibling distance=100pt]
            \tikzstyle{level 2}=[level distance=50pt,sibling distance=120pt]
            \tikzstyle{level 3}=[sibling distance=50pt]
\node[yshift=-30pt](bottom)at (current page.north) {  } child { node(q) {  } child { node(q0cw) {  } edge from parent node[rectangle,draw=green,fill=white,xshift=-5pt,align=center] { $cacao$ \\ $(1,5)$ }   } child { node(q1aw) {  } edge from parent node[rectangle,draw=green,fill=white,xshift=5pt,align=center] { $acao$ \\ $(2,5)$ }   }edge from parent node[rectangle,draw=green,fill=white] { $\Sigma$ }   } ;\draw[style=sll](q) to (bottom);
\node[rectangle,draw=none,yshift=10pt] at (current page.south) {
                    \parbox{.5\textwidth}{
                    \begin{figure}
                        \caption{Suffix Tree for ``$cacao$''}
                    \end{figure}
                    }
            };
        \end{tikzpicture}
    \end{figure}
\end{frame}

    
\begin{frame}[fragile,plain] \frametitle{Suffix Tree}
    \begin{figure}
        \begin{tikzpicture}[remember picture,overlay,level distance=50pt,->,line width=1.5pt]
            \tikzstyle{every node}=[draw,circle,inner sep=1pt,minimum size=7.5pt,line width=0.5pt]
            \tikzstyle{sll}=[bend left,dashed,line width=0.5pt,-{>[length=10pt]}]
            \tikzstyle{slr}=[bend right,dashed,line width=0.5pt,->]
            \tikzstyle{level 1}=[sibling distance=100pt]
            \tikzstyle{level 2}=[level distance=50pt,sibling distance=120pt]
            \tikzstyle{level 3}=[sibling distance=50pt]
\node[yshift=-30pt](bottom)at (current page.north) {  } child { node(q) {  } child { node(q0ca) {  } child { node(q0cw) {  } edge from parent node[rectangle,draw=green,fill=white,align=center] { $cao$ \\ $(3,5)$ }   }edge from parent node[rectangle,draw=green,fill=white,xshift=-5pt,align=center] { $ca$ \\ $(1,2)$ }   } child { node(q1aw) {  } edge from parent node[rectangle,draw=green,fill=white,xshift=5pt,align=center] { $acao$ \\ $(2,5)$ }   }edge from parent node[rectangle,draw=green,fill=white] { $\Sigma$ }   } ;\draw[style=sll](q) to (bottom);
\node[rectangle,draw=none,yshift=10pt] at (current page.south) {
                    \parbox{.5\textwidth}{
                    \begin{figure}
                        \caption{Suffix Tree for ``$cacao$''}
                    \end{figure}
                    }
            };
        \end{tikzpicture}
    \end{figure}
\end{frame}

    
\begin{frame}[fragile,plain] \frametitle{Suffix Tree}
    \begin{figure}
        \begin{tikzpicture}[remember picture,overlay,level distance=50pt,->,line width=1.5pt]
            \tikzstyle{every node}=[draw,circle,inner sep=1pt,minimum size=7.5pt,line width=0.5pt]
            \tikzstyle{sll}=[bend left,dashed,line width=0.5pt,-{>[length=10pt]}]
            \tikzstyle{slr}=[bend right,dashed,line width=0.5pt,->]
            \tikzstyle{level 1}=[sibling distance=100pt]
            \tikzstyle{level 2}=[level distance=50pt,sibling distance=120pt]
            \tikzstyle{level 3}=[sibling distance=50pt]
\node[yshift=-30pt](bottom)at (current page.north) {  } child { node(q) {  } child { node(q0ca) {  } child { node(q0cw) {  } edge from parent node[rectangle,draw=green,fill=white,xshift=-5pt,align=center] { $cao$ \\ $(3,5)$ }   } child { node(q0ca4ow) {  } edge from parent node[rectangle,draw=green,fill=white,xshift=5pt,align=center] { $o$ \\ $(5,5)$ }   }edge from parent node[rectangle,draw=green,fill=white,xshift=-5pt,align=center] { $ca$ \\ $(1,2)$ }   } child { node(q1aw) {  } edge from parent node[rectangle,draw=green,fill=white,xshift=5pt,align=center] { $acao$ \\ $(2,5)$ }   }edge from parent node[rectangle,draw=green,fill=white] { $\Sigma$ }   } ;\draw[style=sll](q) to (bottom);
\node[rectangle,draw=none,yshift=10pt] at (current page.south) {
                    \parbox{.5\textwidth}{
                    \begin{figure}
                        \caption{Suffix Tree for ``$cacao$''}
                    \end{figure}
                    }
            };
        \end{tikzpicture}
    \end{figure}
\end{frame}

    
\begin{frame}[fragile,plain] \frametitle{Suffix Tree}
    \begin{figure}
        \begin{tikzpicture}[remember picture,overlay,level distance=50pt,->,line width=1.5pt]
            \tikzstyle{every node}=[draw,circle,inner sep=1pt,minimum size=7.5pt,line width=0.5pt]
            \tikzstyle{sll}=[bend left,dashed,line width=0.5pt,-{>[length=10pt]}]
            \tikzstyle{slr}=[bend right,dashed,line width=0.5pt,->]
            \tikzstyle{level 1}=[sibling distance=100pt]
            \tikzstyle{level 2}=[level distance=50pt,sibling distance=120pt]
            \tikzstyle{level 3}=[sibling distance=50pt]
\node[yshift=-30pt](bottom)at (current page.north) {  } child { node(q) {  } child { node(q0ca) {  } child { node(q0cw) {  } edge from parent node[rectangle,draw=green,fill=white,xshift=-5pt,align=center] { $cao$ \\ $(3,5)$ }   } child { node(q0ca4ow) {  } edge from parent node[rectangle,draw=green,fill=white,xshift=5pt,align=center] { $o$ \\ $(5,5)$ }   }edge from parent node[rectangle,draw=green,fill=white,xshift=-5pt,align=center] { $ca$ \\ $(1,2)$ }   } child { node(q1a) {  } child { node(q1aw) {  } edge from parent node[rectangle,draw=green,fill=white,align=center] { $cao$ \\ $(3,5)$ }   }edge from parent node[rectangle,draw=green,fill=white,xshift=5pt,align=center] { $a$ \\ $(2,2)$ }   }edge from parent node[rectangle,draw=green,fill=white] { $\Sigma$ }   } ;\draw[style=sll](q) to (bottom);
\node[rectangle,draw=none,yshift=10pt] at (current page.south) {
                    \parbox{.5\textwidth}{
                    \begin{figure}
                        \caption{Suffix Tree for ``$cacao$''}
                    \end{figure}
                    }
            };
        \end{tikzpicture}
    \end{figure}
\end{frame}

    
\begin{frame}[fragile,plain] \frametitle{Suffix Tree}
    \begin{figure}
        \begin{tikzpicture}[remember picture,overlay,level distance=50pt,->,line width=1.5pt]
            \tikzstyle{every node}=[draw,circle,inner sep=1pt,minimum size=7.5pt,line width=0.5pt]
            \tikzstyle{sll}=[bend left,dashed,line width=0.5pt,-{>[length=10pt]}]
            \tikzstyle{slr}=[bend right,dashed,line width=0.5pt,->]
            \tikzstyle{level 1}=[sibling distance=100pt]
            \tikzstyle{level 2}=[level distance=50pt,sibling distance=120pt]
            \tikzstyle{level 3}=[sibling distance=50pt]
\node[yshift=-30pt](bottom)at (current page.north) {  } child { node(q) {  } child { node(q0ca) {  } child { node(q0cw) {  } edge from parent node[rectangle,draw=green,fill=white] { $cao$ }   } child { node(q0ca4ow) {  } edge from parent node[rectangle,draw=green,fill=white] { $o$ }   }edge from parent node[rectangle,draw=green,fill=white] { $ca$ }   } child { node(q1a) {  } child { node(q1a4ow) {  } edge from parent node[rectangle,draw=green,fill=white] { $o$ }   } child { node(q1aw) {  } edge from parent node[rectangle,draw=green,fill=white] { $cao$ }   }edge from parent node[rectangle,draw=green,fill=white] { $a$ }   }edge from parent node[rectangle,draw=green,fill=white] { $\Sigma$ }   } ;\draw[style=sll](q) to (bottom);
\node[rectangle,draw=none,yshift=10pt] at (current page.south) {
                    \parbox{.5\textwidth}{
                    \begin{figure}
                        \caption{Suffix Tree for ``$cacao$''}
                    \end{figure}
                    }
            };
        \end{tikzpicture}
    \end{figure}
\end{frame}

    
\begin{frame}[fragile,plain] \frametitle{Suffix Tree}
    \begin{figure}
        \begin{tikzpicture}[remember picture,overlay,level distance=50pt,->,line width=1.5pt]
            \tikzstyle{every node}=[draw,circle,inner sep=1pt,minimum size=7.5pt,line width=0.5pt]
            \tikzstyle{sll}=[bend left,dashed,line width=0.5pt,-{>[length=10pt]}]
            \tikzstyle{slr}=[bend right,dashed,line width=0.5pt,->]
            \tikzstyle{level 1}=[sibling distance=100pt]
            \tikzstyle{level 2}=[level distance=50pt,sibling distance=120pt]
            \tikzstyle{level 3}=[sibling distance=50pt]
\node[yshift=-30pt](bottom)at (current page.north) {  } child { node(q) {  } child { node(q0ca) {  } child { node(q0cw) {  } edge from parent node[rectangle,draw=green,fill=white] { $cao$ }   } child { node(q0ca4ow) {  } edge from parent node[rectangle,draw=green,fill=white] { $o$ }   }edge from parent node[rectangle,draw=green,fill=white] { $ca$ }   } child { node(q1a) {  } child { node(q1a4ow) {  } edge from parent node[rectangle,draw=green,fill=white] { $o$ }   } child { node(q1aw) {  } edge from parent node[rectangle,draw=green,fill=white] { $cao$ }   }edge from parent node[rectangle,draw=green,fill=white] { $a$ }   }edge from parent node[rectangle,draw=green,fill=white] { $\Sigma$ }   } ;\draw[style=sll](q) to (bottom);\draw[style=sll](q0ca) to (q1a);
\node[rectangle,draw=none,yshift=10pt] at (current page.south) {
                    \parbox{.5\textwidth}{
                    \begin{figure}
                        \caption{Suffix Tree for ``$cacao$''}
                    \end{figure}
                    }
            };
        \end{tikzpicture}
    \end{figure}
\end{frame}

    
\begin{frame}[fragile,plain] \frametitle{Suffix Tree}
    \begin{figure}
        \begin{tikzpicture}[remember picture,overlay,level distance=50pt,->,line width=1.5pt]
            \tikzstyle{every node}=[draw,circle,inner sep=1pt,minimum size=7.5pt,line width=0.5pt]
            \tikzstyle{sll}=[bend left,dashed,line width=0.5pt,-{>[length=10pt]}]
            \tikzstyle{slr}=[bend right,dashed,line width=0.5pt,->]
            \tikzstyle{level 1}=[sibling distance=100pt]
            \tikzstyle{level 2}=[level distance=50pt,sibling distance=120pt]
            \tikzstyle{level 3}=[sibling distance=50pt]
\node[yshift=-30pt](bottom)at (current page.north) {  } child { node(q) {  } child { node(q0ca) {  } child { node(q0cw) {  } edge from parent node[rectangle,draw=green,fill=white] { $cao$ }   } child { node(q0ca4ow) {  } edge from parent node[rectangle,draw=green,fill=white] { $o$ }   }edge from parent node[rectangle,draw=green,fill=white] { $ca$ }   } child { node(q1a) {  } child { node(q1a4ow) {  } edge from parent node[rectangle,draw=green,fill=white] { $o$ }   } child { node(q1aw) {  } edge from parent node[rectangle,draw=green,fill=white] { $cao$ }   }edge from parent node[rectangle,draw=green,fill=white] { $a$ }   } child { node(q4ow) {  } edge from parent node[rectangle,draw=green,fill=white] { $o$ }   }edge from parent node[rectangle,draw=green,fill=white] { $\Sigma$ }   } ;\draw[style=sll](q) to (bottom);\draw[style=sll](q0ca) to (q1a);
\node[rectangle,draw=none,yshift=10pt] at (current page.south) {
                    \parbox{.5\textwidth}{
                    \begin{figure}
                        \caption{Suffix Tree for ``$cacao$''}
                    \end{figure}
                    }
            };
        \end{tikzpicture}
    \end{figure}
\end{frame}

    
\begin{frame}[fragile,plain] \frametitle{Suffix Tree}
    \begin{figure}
        \begin{tikzpicture}[remember picture,overlay,level distance=50pt,->,line width=1.5pt]
            \tikzstyle{every node}=[draw,circle,inner sep=1pt,minimum size=7.5pt,line width=0.5pt]
            \tikzstyle{sll}=[bend left,dashed,line width=0.5pt,-{>[length=10pt]}]
            \tikzstyle{slr}=[bend right,dashed,line width=0.5pt,->]
            \tikzstyle{level 1}=[sibling distance=100pt]
            \tikzstyle{level 2}=[level distance=50pt,sibling distance=120pt]
            \tikzstyle{level 3}=[sibling distance=50pt]
\node[yshift=-30pt](bottom)at (current page.north) {  } child { node(q) {  } child { node(q0ca) {  } child { node(q0cw) {  } edge from parent node[rectangle,draw=green,fill=white,xshift=-5pt,align=center] { $cao$ \\ $(3,5)$ }   } child { node(q0ca4ow) {  } edge from parent node[rectangle,draw=green,fill=white,xshift=5pt,align=center] { $o$ \\ $(5,5)$ }   }edge from parent node[rectangle,draw=green,fill=white,xshift=-10pt,align=center] { $ca$ \\ $(1,2)$ }   } child { node(q1a) {  } child { node(q1a4ow) {  } edge from parent node[rectangle,draw=green,fill=white,xshift=-5pt,align=center] { $o$ \\ $(5,5)$ }   } child { node(q1aw) {  } edge from parent node[rectangle,draw=green,fill=white,xshift=5pt,align=center] { $cao$ \\ $(3,5)$ }   }edge from parent node[rectangle,draw=green,fill=white,align=center] { $a$ \\ $(2,2)$ }   } child { node(q4ow) {  } edge from parent node[rectangle,draw=green,fill=white,xshift=10pt,align=center] { $o$ \\ $(5,5)$ }   }edge from parent node[rectangle,draw=green,fill=white] { $\Sigma$ }   } ;\draw[style=sll](q) to (bottom);\draw[style=sll](q0ca) to (q1a);\draw[style=sll](q1a) to (q);
\node[rectangle,draw=none,yshift=10pt] at (current page.south) {
                    \parbox{.5\textwidth}{
                    \begin{figure}
                        \caption{Suffix Tree for ``$cacao$''}
                    \end{figure}
                    }
            };
        \end{tikzpicture}
    \end{figure}
\end{frame}


    
\begin{frame}[fragile,plain] \frametitle{Suffix Tree}
\centering
    \begin{figure}
        \begin{tikzpicture}[remember picture,overlay,level distance=50pt,->,line width=1.5pt]
            \tikzstyle{every node}=[draw,circle,inner sep=1pt,minimum size=7.5pt,line width=0.5pt]
            \tikzstyle{sll}=[bend left,dashed,line width=0.5pt,-{>[length=10pt]}]
            \tikzstyle{slr}=[bend right,dashed,line width=0.5pt,->]
            \tikzstyle{level 1}=[sibling distance=100pt]
            \tikzstyle{level 2}=[level distance=50pt,sibling distance=120pt]
            \tikzstyle{level 3}=[sibling distance=50pt]
\node[yshift=-30pt](bottom)at (current page.north) {  } child { node(q) {  } edge from parent node[rectangle,draw=green,fill=white] { $\Sigma$ }   } ;\draw[style=sll](q) to (bottom);
\node[rectangle,draw=none,yshift=10pt] at (current page.south) {
                    \parbox{.5\textwidth}{
                    \begin{figure}
                        \caption{Suffix Tree for ``$abbababc$''}
                    \end{figure}
                    }
            };
        \end{tikzpicture}
    \end{figure}
\end{frame}

    
\begin{frame}[fragile,plain] \frametitle{Suffix Tree}
    \begin{figure}
        \begin{tikzpicture}[remember picture,overlay,level distance=50pt,->,line width=1.5pt]
            \tikzstyle{every node}=[draw,circle,inner sep=1pt,minimum size=7.5pt,line width=0.5pt]
            \tikzstyle{sll}=[bend left,dashed,line width=0.5pt,-{>[length=10pt]}]
            \tikzstyle{slr}=[bend right,dashed,line width=0.5pt,->]
            \tikzstyle{level 1}=[sibling distance=100pt]
            \tikzstyle{level 2}=[level distance=50pt,sibling distance=120pt]
            \tikzstyle{level 3}=[sibling distance=50pt]
\node[yshift=-30pt](bottom)at (current page.north) {  } child { node(q) {  } child { node(q0aw) {  } edge from parent node[rectangle,draw=green,fill=white,align=center] { $abbababc$ \\ $(1,8)$ }   }edge from parent node[rectangle,draw=green,fill=white] { $\Sigma$ }   } ;\draw[style=sll](q) to (bottom);
\node[rectangle,draw=none,yshift=10pt] at (current page.south) {
                    \parbox{.5\textwidth}{
                    \begin{figure}
                        \caption{Suffix Tree for ``$abbababc$''}
                    \end{figure}
                    }
            };
        \end{tikzpicture}
    \end{figure}
\end{frame}

    
\begin{frame}[fragile,plain] \frametitle{Suffix Tree}
\centering
    \begin{figure}
        \begin{tikzpicture}[remember picture,overlay,level distance=50pt,->,line width=1.5pt]
            \tikzstyle{every node}=[draw,circle,inner sep=1pt,minimum size=7.5pt,line width=0.5pt]
            \tikzstyle{sll}=[bend left,dashed,line width=0.5pt,-{>[length=10pt]}]
            \tikzstyle{slr}=[bend right,dashed,line width=0.5pt,->]
            \tikzstyle{level 1}=[sibling distance=100pt]
            \tikzstyle{level 2}=[level distance=50pt,sibling distance=120pt]
            \tikzstyle{level 3}=[sibling distance=50pt]
\node[yshift=-30pt](bottom)at (current page.north) {  } child { node(q) {  } child { node(q0aw) {  } edge from parent node[rectangle,draw=green,fill=white,xshift=-5pt,align=center] { $abbababc$ \\ $(1,8)$ }   } child { node(q1bw) {  } edge from parent node[rectangle,draw=green,fill=white,xshift=5pt,align=center] { $bbababc$ \\ $(2,8)$ }   }edge from parent node[rectangle,draw=green,fill=white] { $\Sigma$ }   } ;\draw[style=sll](q) to (bottom);
\node[rectangle,draw=none,yshift=10pt] at (current page.south) {
                    \parbox{.5\textwidth}{
                    \begin{figure}
                        \caption{Suffix Tree for ``$abbababc$''}
                    \end{figure}
                    }
            };
        \end{tikzpicture}
    \end{figure}
\end{frame}

    
\begin{frame}[fragile,plain] \frametitle{Suffix Tree}
    \begin{figure}
        \begin{tikzpicture}[remember picture,overlay,level distance=50pt,->,line width=1.5pt]
            \tikzstyle{every node}=[draw,circle,inner sep=1pt,minimum size=7.5pt,line width=0.5pt]
            \tikzstyle{sll}=[bend left,dashed,line width=0.5pt,-{>[length=10pt]}]
            \tikzstyle{slr}=[bend right,dashed,line width=0.5pt,->]
            \tikzstyle{level 1}=[sibling distance=100pt]
            \tikzstyle{level 2}=[level distance=50pt,sibling distance=120pt]
            \tikzstyle{level 3}=[sibling distance=50pt]
\node[yshift=-30pt](bottom)at (current page.north) {  } child { node(q) {  } child { node(q0aw) {  } edge from parent node[rectangle,draw=green,fill=white] { $abbababc$ }   } child { node(q1b) {  } child { node(q1bw) {  } edge from parent node[rectangle,draw=green,fill=white] { $bababc$ }   }edge from parent node[rectangle,draw=green,fill=white] { $b$ }   }edge from parent node[rectangle,draw=green,fill=white] { $\Sigma$ }   } ;\draw[style=sll](q) to (bottom);
\node[rectangle,draw=none,yshift=10pt] at (current page.south) {
                    \parbox{.5\textwidth}{
                    \begin{figure}
                        \caption{Suffix Tree for ``$abbababc$''}
                    \end{figure}
                    }
            };
        \end{tikzpicture}
    \end{figure}
\end{frame}

    
\begin{frame}[fragile,plain] \frametitle{Suffix Tree}
    \begin{figure}
        \begin{tikzpicture}[remember picture,overlay,level distance=50pt,->,line width=1.5pt]
            \tikzstyle{every node}=[draw,circle,inner sep=1pt,minimum size=7.5pt,line width=0.5pt]
            \tikzstyle{sll}=[bend left,dashed,line width=0.5pt,-{>[length=10pt]}]
            \tikzstyle{slr}=[bend right,dashed,line width=0.5pt,->]
            \tikzstyle{level 1}=[sibling distance=100pt]
            \tikzstyle{level 2}=[level distance=50pt,sibling distance=120pt]
            \tikzstyle{level 3}=[sibling distance=50pt]
\node[yshift=-30pt](bottom)at (current page.north) {  } child { node(q) {  } child { node(q0aw) {  } edge from parent node[rectangle,draw=green,fill=white,xshift=-5pt,align=center] { $abbababc$ \\ $(1,8)$ }   } child { node(q1b) {  } child { node(q1b3aw) {  } edge from parent node[rectangle,draw=green,fill=white,xshift=-5pt,align=center] { $ababc$ \\ $(4,8)$ }   } child { node(q1bw) {  } edge from parent node[rectangle,draw=green,fill=white,xshift=5pt,align=center] { $bababc$ \\ $(3,8)$ }   }edge from parent node[rectangle,draw=green,fill=white,xshift=5pt,align=center] { $b$ \\ $(2,2)$ }   }edge from parent node[rectangle,draw=green,fill=white] { $\Sigma$ }   } ;\draw[style=sll](q) to (bottom);
\node[rectangle,draw=none,yshift=10pt] at (current page.south) {
                    \parbox{.5\textwidth}{
                    \begin{figure}
                        \caption{Suffix Tree for ``$abbababc$''}
                    \end{figure}
                    }
            };
        \end{tikzpicture}
    \end{figure}
\end{frame}

    
\begin{frame}[fragile,plain] \frametitle{Suffix Tree}
    \begin{figure}
        \begin{tikzpicture}[remember picture,overlay,level distance=50pt,->,line width=1.5pt]
            \tikzstyle{every node}=[draw,circle,inner sep=1pt,minimum size=7.5pt,line width=0.5pt]
            \tikzstyle{sll}=[bend left,dashed,line width=0.5pt,-{>[length=10pt]}]
            \tikzstyle{slr}=[bend right,dashed,line width=0.5pt,->]
            \tikzstyle{level 1}=[sibling distance=100pt]
            \tikzstyle{level 2}=[level distance=50pt,sibling distance=120pt]
            \tikzstyle{level 3}=[sibling distance=50pt]
\node[yshift=-30pt](bottom)at (current page.north) {  } child { node(q) {  } child { node(q0aw) {  } edge from parent node[rectangle,draw=green,fill=white] { $abbababc$ }   } child { node(q1b) {  } child { node(q1b3aw) {  } edge from parent node[rectangle,draw=green,fill=white] { $ababc$ }   } child { node(q1bw) {  } edge from parent node[rectangle,draw=green,fill=white] { $bababc$ }   }edge from parent node[rectangle,draw=green,fill=white] { $b$ }   }edge from parent node[rectangle,draw=green,fill=white] { $\Sigma$ }   } ;\draw[style=sll](q) to (bottom);\draw[style=sll](q1b) to (q);
\node[rectangle,draw=none,yshift=10pt] at (current page.south) {
                    \parbox{.5\textwidth}{
                    \begin{figure}
                        \caption{Suffix Tree for ``$abbababc$''}
                    \end{figure}
                    }
            };
        \end{tikzpicture}
    \end{figure}
\end{frame}

    
\begin{frame}[fragile,plain] \frametitle{Suffix Tree}
    \begin{figure}
        \begin{tikzpicture}[remember picture,overlay,level distance=50pt,->,line width=1.5pt]
            \tikzstyle{every node}=[draw,circle,inner sep=1pt,minimum size=7.5pt,line width=0.5pt]
            \tikzstyle{sll}=[bend left,dashed,line width=0.5pt,-{>[length=10pt]}]
            \tikzstyle{slr}=[bend right,dashed,line width=0.5pt,->]
            \tikzstyle{level 1}=[sibling distance=100pt]
            \tikzstyle{level 2}=[level distance=50pt,sibling distance=120pt]
            \tikzstyle{level 3}=[sibling distance=50pt]
\node[yshift=-30pt](bottom)at (current page.north) {  } child { node(q) {  } child { node(q0ab) {  } child { node(q0aw) {  } edge from parent node[rectangle,draw=green,fill=white] { $bababc$ }   }edge from parent node[rectangle,draw=green,fill=white] { $ab$ }   } child { node(q1b) {  } child { node(q1b3aw) {  } edge from parent node[rectangle,draw=green,fill=white] { $ababc$ }   } child { node(q1bw) {  } edge from parent node[rectangle,draw=green,fill=white] { $bababc$ }   }edge from parent node[rectangle,draw=green,fill=white] { $b$ }   }edge from parent node[rectangle,draw=green,fill=white] { $\Sigma$ }   } ;\draw[style=sll](q) to (bottom);\draw[style=sll](q1b) to (q);
\node[rectangle,draw=none,yshift=10pt] at (current page.south) {
                    \parbox{.5\textwidth}{
                    \begin{figure}
                        \caption{Suffix Tree for ``$abbababc$''}
                    \end{figure}
                    }
            };
        \end{tikzpicture}
    \end{figure}
\end{frame}

    
\begin{frame}[fragile,plain] \frametitle{Suffix Tree}
    \begin{figure}
        \begin{tikzpicture}[remember picture,overlay,level distance=50pt,->,line width=1.5pt]
            \tikzstyle{every node}=[draw,circle,inner sep=1pt,minimum size=7.5pt,line width=0.5pt]
            \tikzstyle{sll}=[bend left,dashed,line width=0.5pt,-{>[length=10pt]}]
            \tikzstyle{slr}=[bend right,dashed,line width=0.5pt,->]
            \tikzstyle{level 1}=[sibling distance=100pt]
            \tikzstyle{level 2}=[level distance=50pt,sibling distance=120pt]
            \tikzstyle{level 3}=[sibling distance=50pt]
\node[yshift=-30pt](bottom)at (current page.north) {  } child { node(q) {  } child { node(q0ab) {  } child { node(q0aw) {  } edge from parent node[rectangle,draw=green,fill=white] { $bababc$ }   } child { node(q0ab5aw) {  } edge from parent node[rectangle,draw=green,fill=white] { $abc$ }   }edge from parent node[rectangle,draw=green,fill=white] { $ab$ }   } child { node(q1b) {  } child { node(q1b3aw) {  } edge from parent node[rectangle,draw=green,fill=white] { $ababc$ }   } child { node(q1bw) {  } edge from parent node[rectangle,draw=green,fill=white] { $bababc$ }   }edge from parent node[rectangle,draw=green,fill=white] { $b$ }   }edge from parent node[rectangle,draw=green,fill=white] { $\Sigma$ }   } ;\draw[style=sll](q) to (bottom);\draw[style=sll](q1b) to (q);
\node[rectangle,draw=none,yshift=10pt] at (current page.south) {
                    \parbox{.5\textwidth}{
                    \begin{figure}
                        \caption{Suffix Tree for ``$abbababc$''}
                    \end{figure}
                    }
            };
        \end{tikzpicture}
    \end{figure}
\end{frame}

    
\begin{frame}[fragile,plain] \frametitle{Suffix Tree}
    \begin{figure}
        \begin{tikzpicture}[remember picture,overlay,level distance=50pt,->,line width=1.5pt]
            \tikzstyle{every node}=[draw,circle,inner sep=1pt,minimum size=7.5pt,line width=0.5pt]
            \tikzstyle{sll}=[bend left,dashed,line width=0.5pt,-{>[length=10pt]}]
            \tikzstyle{slr}=[bend right,dashed,line width=0.5pt,->]
            \tikzstyle{level 1}=[sibling distance=100pt]
            \tikzstyle{level 2}=[level distance=50pt,sibling distance=120pt]
            \tikzstyle{level 3}=[sibling distance=50pt]
\node[yshift=-30pt](bottom)at (current page.north) {  } child { node(q) {  } child { node(q0ab) {  } child { node(q0aw) {  } edge from parent node[rectangle,draw=green,fill=white,xshift=-5pt,align=center] { $bababc$ \\ $(3,8)$ }   } child { node(q0ab5aw) {  } edge from parent node[rectangle,draw=green,fill=white,xshift=5pt,align=center] { $abc$ \\ $(6,8)$ }   }edge from parent node[rectangle,draw=green,fill=white,xshift=-5pt,align=center] { $ab$ \\ $(1,2)$ }   } child { node(q1b) {  } child { node(q1b3aw) {  } edge from parent node[rectangle,draw=green,fill=white,xshift=-5pt,align=center] { $ababc$ \\ $(4,8)$ }   } child { node(q1bw) {  } edge from parent node[rectangle,draw=green,fill=white,xshift=5pt,align=center] { $bababc$ \\ $(3,8)$ }   }edge from parent node[rectangle,draw=green,fill=white,xshift=5pt,align=center] { $b$ \\ $(2,2)$ }   }edge from parent node[rectangle,draw=green,fill=white] { $\Sigma$ }   } ;\draw[style=sll](q) to (bottom);\draw[style=sll](q1b) to (q);\draw[style=sll](q0ab) to (q1b);
\node[rectangle,draw=none,yshift=10pt] at (current page.south) {
                    \parbox{.5\textwidth}{
                    \begin{figure}
                        \caption{Suffix Tree for ``$abbababc$''}
                    \end{figure}
                    }
            };
        \end{tikzpicture}
    \end{figure}
\end{frame}

    
\begin{frame}[fragile,plain] \frametitle{Suffix Tree}
    \begin{figure}
        \begin{tikzpicture}[remember picture,overlay,level distance=50pt,->,line width=1.5pt]
            \tikzstyle{every node}=[draw,circle,inner sep=1pt,minimum size=7.5pt,line width=0.5pt]
            \tikzstyle{sll}=[bend left,dashed,line width=0.5pt,-{>[length=10pt]}]
            \tikzstyle{slr}=[bend right,dashed,line width=0.5pt,->]
            \tikzstyle{level 1}=[sibling distance=100pt]
            \tikzstyle{level 2}=[level distance=50pt,sibling distance=120pt]
            \tikzstyle{level 3}=[sibling distance=50pt]
\node[yshift=-30pt](bottom)at (current page.north) {  } child { node(q) {  } child { node(q0ab) {  } child { node(q0aw) {  } edge from parent node[rectangle,draw=green,fill=white] { $bababc$ }   } child { node(q0ab5aw) {  } edge from parent node[rectangle,draw=green,fill=white] { $abc$ }   }edge from parent node[rectangle,draw=green,fill=white] { $ab$ }   } child { node(q1b) {  } child { node(q1b3ab) {  } child { node(q1b3aw) {  } edge from parent node[rectangle,draw=green,fill=white] { $abc$ }   }edge from parent node[rectangle,draw=green,fill=white] { $ab$ }   } child { node(q1bw) {  } edge from parent node[rectangle,draw=green,fill=white] { $bababc$ }   }edge from parent node[rectangle,draw=green,fill=white] { $b$ }   }edge from parent node[rectangle,draw=green,fill=white] { $\Sigma$ }   } ;\draw[style=sll](q) to (bottom);\draw[style=sll](q1b) to (q);\draw[style=sll](q0ab) to (q1b);
\node[rectangle,draw=none,yshift=10pt] at (current page.south) {
                    \parbox{.5\textwidth}{
                    \begin{figure}
                        \caption{Suffix Tree for ``$abbababc$''}
                    \end{figure}
                    }
            };
        \end{tikzpicture}
    \end{figure}
\end{frame}

    
\begin{frame}[fragile,plain] \frametitle{Suffix Tree}
    \begin{figure}
        \begin{tikzpicture}[remember picture,overlay,level distance=50pt,->,line width=1.5pt]
            \tikzstyle{every node}=[draw,circle,inner sep=1pt,minimum size=7.5pt,line width=0.5pt]
            \tikzstyle{sll}=[bend left,dashed,line width=0.5pt,-{>[length=10pt]}]
            \tikzstyle{slr}=[bend right,dashed,line width=0.5pt,->]
            \tikzstyle{level 1}=[sibling distance=100pt]
            \tikzstyle{level 2}=[level distance=50pt,sibling distance=120pt]
            \tikzstyle{level 3}=[sibling distance=50pt]
\node[yshift=-30pt](bottom)at (current page.north) {  } child { node(q) {  } child { node(q0ab) {  } child { node(q0aw) {  } edge from parent node[rectangle,draw=green,fill=white,xshift=-5pt,align=center] { $bababc$ \\ $(3,8)$ }   } child { node(q0ab5aw) {  } edge from parent node[rectangle,draw=green,fill=white,xshift=5pt,align=center] { $abc$ \\ $(6,8)$ }   }edge from parent node[rectangle,draw=green,fill=white,xshift=-5pt,align=center] { $ab$ \\ $(1,2)$ }   } child { node(q1b) {  } child { node(q1b3ab) {  } child { node(q1b3aw) {  } edge from parent node[rectangle,draw=green,fill=white,xshift=-5pt,align=center] { $abc$ \\ $(6,8)$ }   } child { node(q1b3ab7cw) {  } edge from parent node[rectangle,draw=green,fill=white,xshift=5pt,align=center] { $c$ \\ $(8,8)$ }   }edge from parent node[rectangle,draw=green,fill=white,xshift=-5pt,align=center] { $ab$ \\ $(4,5)$ }   } child { node(q1bw) {  } edge from parent node[rectangle,draw=green,fill=white,xshift=5pt,align=center] { $bababc$ \\ $(3,8)$ }   }edge from parent node[rectangle,draw=green,fill=white,xshift=5pt,align=center] { $b$ \\ $(2,2)$ }   }edge from parent node[rectangle,draw=green,fill=white] { $\Sigma$ }   } ;\draw[style=sll](q) to (bottom);\draw[style=sll](q1b) to (q);\draw[style=sll](q0ab) to (q1b);
\node[rectangle,draw=none,yshift=10pt] at (current page.south) {
                    \parbox{.5\textwidth}{
                    \begin{figure}
                        \caption{Suffix Tree for ``$abbababc$''}
                    \end{figure}
                    }
            };
        \end{tikzpicture}
    \end{figure}
\end{frame}

    
\begin{frame}[fragile,plain] \frametitle{Suffix Tree}
    \begin{figure}
        \begin{tikzpicture}[remember picture,overlay,level distance=50pt,->,line width=1.5pt]
            \tikzstyle{every node}=[draw,circle,inner sep=1pt,minimum size=7.5pt,line width=0.5pt]
            \tikzstyle{sll}=[bend left,dashed,line width=0.5pt,-{>[length=10pt]}]
            \tikzstyle{slr}=[bend right,dashed,line width=0.5pt,->]
            \tikzstyle{level 1}=[sibling distance=100pt]
            \tikzstyle{level 2}=[level distance=50pt,sibling distance=120pt]
            \tikzstyle{level 3}=[sibling distance=50pt]
\node[yshift=-30pt](bottom)at (current page.north) {  } child { node(q) {  } child { node(q0ab) {  } child { node(q0aw) {  } edge from parent node[rectangle,draw=green,fill=white] { $bababc$ }   } child { node(q0ab5aw) {  } edge from parent node[rectangle,draw=green,fill=white] { $abc$ }   } child { node(q0ab7cw) {  } edge from parent node[rectangle,draw=green,fill=white] { $c$ }   }edge from parent node[rectangle,draw=green,fill=white] { $ab$ }   } child { node(q1b) {  } child { node(q1b3ab) {  } child { node(q1b3aw) {  } edge from parent node[rectangle,draw=green,fill=white] { $abc$ }   } child { node(q1b3ab7cw) {  } edge from parent node[rectangle,draw=green,fill=white] { $c$ }   }edge from parent node[rectangle,draw=green,fill=white] { $ab$ }   } child { node(q1bw) {  } edge from parent node[rectangle,draw=green,fill=white] { $bababc$ }   }edge from parent node[rectangle,draw=green,fill=white] { $b$ }   }edge from parent node[rectangle,draw=green,fill=white] { $\Sigma$ }   } ;\draw[style=sll](q) to (bottom);\draw[style=sll](q1b) to (q);\draw[style=sll](q0ab) to (q1b);
\node[rectangle,draw=none,yshift=10pt] at (current page.south) {
                    \parbox{.5\textwidth}{
                    \begin{figure}
                        \caption{Suffix Tree for ``$abbababc$''}
                    \end{figure}
                    }
            };
        \end{tikzpicture}
    \end{figure}
\end{frame}

    
\begin{frame}[fragile,plain] \frametitle{Suffix Tree}
    \begin{figure}
        \begin{tikzpicture}[remember picture,overlay,level distance=50pt,->,line width=1.5pt]
            \tikzstyle{every node}=[draw,circle,inner sep=1pt,minimum size=7.5pt,line width=0.5pt]
            \tikzstyle{sll}=[bend left,dashed,line width=0.5pt,-{>[length=10pt]}]
            \tikzstyle{slr}=[bend right,dashed,line width=0.5pt,->]
            \tikzstyle{level 1}=[sibling distance=100pt]
            \tikzstyle{level 2}=[level distance=50pt,sibling distance=120pt]
            \tikzstyle{level 3}=[sibling distance=50pt]
\node[yshift=-30pt](bottom)at (current page.north) {  } child { node(q) {  } child { node(q0ab) {  } child { node(q0aw) {  } edge from parent node[rectangle,draw=green,fill=white,xshift=-10pt,align=center] { $bababc$ \\ $(3,8)$ }   } child { node(q0ab5aw) {  } edge from parent node[rectangle,draw=green,fill=white,align=center] { $abc$ \\ $(6,8)$ }   } child { node(q0ab7cw) {  } edge from parent node[rectangle,draw=green,fill=white,xshift=10pt,align=center] { $c$ \\ $(8,8)$ }   }edge from parent node[rectangle,draw=green,fill=white,xshift=-5pt,align=center] { $ab$ \\ $(1,2)$ }   } child { node(q1b) {  } child { node(q1b3ab) {  } child { node(q1b3aw) {  } edge from parent node[rectangle,draw=green,fill=white,xshift=-5pt,align=center] { $abc$ \\ $(6,8)$ }   } child { node(q1b3ab7cw) {  } edge from parent node[rectangle,draw=green,fill=white,xshift=5pt,align=center] { $c$ \\ $(8,8)$ }   }edge from parent node[rectangle,draw=green,fill=white,xshift=-5pt,align=center] { $ab$ \\ $(4,5)$ }   } child { node(q1bw) {  } edge from parent node[rectangle,draw=green,fill=white,xshift=5pt,align=center] { $bababc$ \\ $(3,8)$ }   }edge from parent node[rectangle,draw=green,fill=white,xshift=5pt,align=center] { $b$ \\ $(2,2)$ }   }edge from parent node[rectangle,draw=green,fill=white] { $\Sigma$ }   } ;\draw[style=sll](q) to (bottom);\draw[style=sll](q1b) to (q);\draw[style=sll](q0ab) to (q1b);\draw[style=sll](q1b3ab) to (q0ab);
\node[rectangle,draw=none,yshift=10pt] at (current page.south) {
                    \parbox{.5\textwidth}{
                    \begin{figure}
                        \caption{Suffix Tree for ``$abbababc$''}
                    \end{figure}
                    }
            };
        \end{tikzpicture}
    \end{figure}
\end{frame}

    
\begin{frame}[fragile,plain] \frametitle{Suffix Tree}
    \begin{figure}
        \begin{tikzpicture}[remember picture,overlay,level distance=50pt,->,line width=1.5pt]
            \tikzstyle{every node}=[draw,circle,inner sep=1pt,minimum size=7.5pt,line width=0.5pt]
            \tikzstyle{sll}=[bend left,dashed,line width=0.5pt,-{>[length=10pt]}]
            \tikzstyle{slr}=[bend right,dashed,line width=0.5pt,->]
            \tikzstyle{level 1}=[sibling distance=100pt]
            \tikzstyle{level 2}=[level distance=50pt,sibling distance=120pt]
            \tikzstyle{level 3}=[sibling distance=50pt]
\node[yshift=-30pt](bottom)at (current page.north) {  } child { node(q) {  } child { node(q0ab) {  } child { node(q0aw) {  } edge from parent node[rectangle,draw=green,fill=white] { $bababc$ }   } child { node(q0ab5aw) {  } edge from parent node[rectangle,draw=green,fill=white] { $abc$ }   } child { node(q0ab7cw) {  } edge from parent node[rectangle,draw=green,fill=white] { $c$ }   }edge from parent node[rectangle,draw=green,fill=white] { $ab$ }   } child { node(q1b) {  } child { node(q1b3ab) {  } child { node(q1b3aw) {  } edge from parent node[rectangle,draw=green,fill=white] { $abc$ }   } child { node(q1b3ab7cw) {  } edge from parent node[rectangle,draw=green,fill=white] { $c$ }   }edge from parent node[rectangle,draw=green,fill=white] { $ab$ }   } child { node(q1b7cw) {  } edge from parent node[rectangle,draw=green,fill=white] { $c$ }   } child { node(q1bw) {  } edge from parent node[rectangle,draw=green,fill=white] { $bababc$ }   }edge from parent node[rectangle,draw=green,fill=white] { $b$ }   }edge from parent node[rectangle,draw=green,fill=white] { $\Sigma$ }   } ;\draw[style=sll](q) to (bottom);\draw[style=sll](q1b) to (q);\draw[style=sll](q0ab) to (q1b);\draw[style=sll](q1b3ab) to (q0ab);
\node[rectangle,draw=none,yshift=10pt] at (current page.south) {
                    \parbox{.5\textwidth}{
                    \begin{figure}
                        \caption{Suffix Tree for ``$abbababc$''}
                    \end{figure}
                    }
            };
        \end{tikzpicture}
    \end{figure}
\end{frame}

    
\begin{frame}[fragile,plain] \frametitle{Suffix Tree}
\centering
    \begin{figure}
        \begin{tikzpicture}[remember picture,overlay,level distance=50pt,->,line width=1.5pt]
            \tikzstyle{every node}=[draw,circle,inner sep=1pt,minimum size=7.5pt,line width=0.5pt]
            \tikzstyle{sll}=[bend left,dashed,line width=0.5pt,-{>[length=10pt]}]
            \tikzstyle{slr}=[bend right,dashed,line width=0.5pt,->]
            \tikzstyle{level 1}=[sibling distance=100pt]
            \tikzstyle{level 2}=[level distance=50pt,sibling distance=120pt]
            \tikzstyle{level 3}=[sibling distance=50pt]
\node[yshift=-30pt](bottom)at (current page.north) {  } child { node(q) {  } child { node(q0ab) {  } child { node(q0aw) {  } edge from parent node[rectangle,draw=green,fill=white,xshift=-10pt,align=center] { $bababc$ \\ $(3,8)$ }   } child { node(q0ab5aw) {  } edge from parent node[rectangle,draw=green,fill=white,align=center] { $abc$ \\ $(6,8)$ }   } child { node(q0ab7cw) {  } edge from parent node[rectangle,draw=green,fill=white,xshift=10pt,align=center] { $c$ \\ $(8,8)$ }   }edge from parent node[rectangle,draw=green,fill=white,xshift=-10pt,align=center] { $ab$ \\ $(1,2)$ }   } child { node(q1b) {  } child { node(q1b3ab) {  } child { node(q1b3aw) {  } edge from parent node[rectangle,draw=green,fill=white,xshift=-5pt,align=center] { $abc$ \\ $(6,8)$ }   } child { node(q1b3ab7cw) {  } edge from parent node[rectangle,draw=green,fill=white,xshift=5pt,align=center] { $c$ \\ $(8,8)$ }   }edge from parent node[rectangle,draw=green,fill=white,xshift=-10pt,align=center] { $ab$ \\ $(4,5)$ }   } child { node(q1b7cw) {  } edge from parent node[rectangle,draw=green,fill=white,align=center] { $c$ \\ $(8,8)$ }   } child { node(q1bw) {  } edge from parent node[rectangle,draw=green,fill=white,xshift=10pt,align=center] { $bababc$ \\ $(3,8)$ }   }edge from parent node[rectangle,draw=green,fill=white,align=center] { $b$ \\ $(2,2)$ }   } child { node(q7cw) {  } edge from parent node[rectangle,draw=green,fill=white,xshift=10pt,align=center] { $c$ \\ $(8,8)$ }   }edge from parent node[rectangle,draw=green,fill=white] { $\Sigma$ }   } ;\draw[style=sll](q) to (bottom);\draw[style=sll](q1b) to (q);\draw[style=sll](q0ab) to (q1b);\draw[style=sll](q1b3ab) to (q0ab);
\node[rectangle,draw=none,yshift=10pt] at (current page.south) {
                    \parbox{.5\textwidth}{
                    \begin{figure}
                        \caption{Suffix Tree for ``$abbababc$''}
                    \end{figure}
                    }
            };
        \end{tikzpicture}
    \end{figure}
\end{frame}


    \begin{frame}[fragile,plain]\frametitle{Construction of Suffix Tree}
        \begin{definitionblock}{Explicit states}
            $Q' \cup \{\bot\}$ is the \underline{explicit states} of $STrie(T)$ \\
            \hfill \break
            $Q' \subseteq Q$ consists of all branching states and all leaves of $STrie(T)$ \\
            By definition, $root$ is included into the branching states
        \end{definitionblock}
        \begin{definitionblock}{Implicit states}
            Other states of $STrie(T)$ is the \underline{implicit states}
        \end{definitionblock}
    \end{frame}

    \begin{frame}[fragile,plain]\frametitle{Construction of Suffix Tree}
        \begin{definitionblock}{Generalized transition}
            $g'(s, w) = r$ in $STree(T)$ represents the string $w$ spelled out by the transition path in $STrie(T)$ between two explicit states $s$ and $r$ \\
            \hfill \break
            To save space, the string $w$ is represented as a pair $(k, p)$ of pointers: $t_k\cdots t_p = w$ \\
            Then $g'(s, (k, p)) = r$
        \end{definitionblock}
        Such pointers exist because there must be a suffix $T_i$ s.t. the transition path for $T_i$ in $STrie(T)$ goes through $s$ and $r$ \\
        \hfill \break
        \alert{Select the smallest such $i$, and let $k$ and $p$ point to the substring of this $T_i$ that is spelled out by the transition path from $s$ to $r$}
    \end{frame}

    \begin{frame}[fragile,plain]\frametitle{Construction of Suffix Tree}
        \begin{definitionblock}{$a$-transition}
            A transition $g'(s, (k, p)) = r$ is called an \underline{$a$-transition} if $t_k = a$. 
        \end{definitionblock}
        Each $s$ can have at most one $a$-transition $\forall a \in \Sigma$. \\
        \hfill \break
        Let $\Sigma = \{a_1, a_2, \ldots, a_m \}$. \\
        $g(\bot, a_j) = root$ is represented as $g(\bot, (-j, -j)) = root $ for $ j = 1, \ldots, m$.\\
        \hfill \break
        Hence suffix tree $STree(T)$ has two components: The tree itself and the string $T$.
    \end{frame}

    \begin{frame}[fragile,plain]\frametitle{Construction of Suffix Tree}
        $STree(T)$ is of linear size in $|T|$. \\
        $\because Q'$ has at most $|T|$ leaves (at most $1$ leaf for each nonempty suffix) \\
        $\Rightarrow Q'$ has to contain at most $|T| - 1$ branching states (when $|T| > 1$). \\
        \hfill \break
        $\therefore$ There can be at most $2|T| - 2$ transitions between the states in $Q'$ , each taking a constant space because of using pointers instead of an explicit string. \\
        \alert{$\Rightarrow$ In implementation, $g'$ can take $\mathcal{O}(|T|)$ space}
    \end{frame}

    \begin{frame}[fragile,plain]\frametitle{Construction of Suffix Tree}
        \begin{definitionblock}{Suffix function $f'$}
            Let $B \subset Q$ be the set of branching states in $STrie(T)$ \\
            \hfill \break
            $\forall \bar{x} \in B,$ \\
            $
            \begin{cases}
                f'(\bar{x}) = \bar{y} & \mbox{if} \ \bar{x} \neq root, \mbox{then}\ x = ay, a \in \Sigma, \bar{y} \in B \\
                f'(root) = \bot
            \end{cases}
            $
        \end{definitionblock}
        $f'$ is well-defined $\because \bar{x} \in B \Rightarrow f'(\bar{x}) \in B$. \\
        These suffix links are explicitly represented. \underline{Implicit suffix links} are helpful but they are imaginary.
    \end{frame}

    \begin{frame}[fragile,plain]\frametitle{Construction of Suffix Tree}
        \begin{definitionblock}{Suffix Tree}
            Denote suffix tree of $T$ as $STree(T) = (Q' \cup \{\bot\}, root, g', f')$
        \end{definitionblock}
    \end{frame}

    \begin{frame}[fragile,plain]\frametitle{Construction of Suffix Tree}
        \begin{definitionblock}{Reference pair}
            $r = (s, w)$ \\
            Refer to an \alert{explicit or implicit} state $r$ of a suffix tree by a \underline{reference pair} $(s, w)$ where \\
            $s$ is some \alert{explicit} state that is an ancestor of $r$ and \\
            $w$ is the string spelled out by the transitions from $s$ to $r$ in the corresponding suffix trie
        \end{definitionblock}
        \begin{definitionblock}{Canonical reference pair}
            A reference pair is \underline{canonical} if $s$ is the closest ancestor of $r$ (and hence, $w$ is shortest possible)
        \end{definitionblock}
        If $r$ is explicit, canonical reference pair of $r = (r, \epsilon)$
    \end{frame}

    \begin{frame}[fragile,plain]\frametitle{Construction of Suffix Tree}
        Again, represent string $w$ as a pair $(k, p)$ of pointers s.t. $t_k \cdots t_p = w$. \\
        Then, reference pair $(s, w)$ gets form $(s, (k, p))$. $(s, \epsilon) = (s, (p + 1, p))$ \\
        \begin{alertblock}{Caution}
            No contraints on $k$ and $p$ as long as $w$ spells the string
        \end{alertblock}
    \end{frame}

    \begin{frame}[fragile,plain]\frametitle{Construction of Suffix Tree}
        It is technically convenient to omit the final states in the definition of a suffix tree. \\
        When final states are necessary, either \\
        \begin{itemize}
            \item add a symbol $\sharp$ which doesn't occurs in $T$ at the end of $T$ or 
            \item traverse from leaf $\bar{T}$ to $root$ and make all the states on the path explicit
        \end{itemize}
        In many applications of $STree(T)$, the start location of each suffix is stored with the corresponding state. Such an augmented tree can be used as an index for finding any substring of $T$.
    \end{frame}

    \begin{frame}[fragile,plain]\frametitle{Construction of Suffix Tree}
        The algorithm for constructing $STree(T)$ will be patterned after Algorithm~\ref{alg:1}. \\
        Now, we make precise what Algorithm~\ref{alg:1} does. \\
        \hfill \break
        Let $s_1 = \overline{t_1\cdots t_{i-1}},s_2=\overline{t_2\cdots t_{i-1}},s_3,\ldots,s_i=root,s_{i+1}=\bot$ be the states of $STrie(T^{i-1})$ on the \underline{boundary path}. \\
        \hfill \break
        Let $j$ be the smallest index s.t. $s_j$ is not a leaf, and \\
        let $j'$ be the smallest index s.t. $s_{j'}$ has a $t_i$-transition. \\
        \hfill \break
        As $s_1$ is a leaf and $\bot$ is a non-leaf that has a $t_i$-transition, both $j$ and $j'$ are well-defined and $j \leq j'$
    \end{frame}

    \begin{frame}[fragile,plain]\frametitle{Construction of Suffix Tree}
        \begin{theoremblock}{Lemma 1}
            Algorithm~\ref{alg:1} adds to $STrie(T^{i-1})$ a $t_i$-transition $\forall s_h, 1 \leq h < j'$, s.t. \\
            \begin{itemize}
                \item for $1 \leq h < j$, \\
                    the new transition expands an old branch of the trie that ends at leaf $s_h$, \\
                \item and for $j \leq h < j'$, the new transition initiates a new branch from $s_h$. \\
            \end{itemize}
            Algorithm~\ref{alg:1} does not create any other transitions
        \end{theoremblock}
    \end{frame}

    \begin{frame}[fragile,plain]\frametitle{Construction of Suffix Tree}
        \begin{definitionblock}{Active point}
            $s_j$ is the \underline{active point} of $\alert{STrie}(T^{i-1})$
        \end{definitionblock}
        \begin{definitionblock}{End point}
            $s_{j'}$ is the \underline{end point} of $\alert{STrie}(T^{i-1})$
        \end{definitionblock}
        These states are present, explicitly or implicitly, in $STree(T^{i-1})$ \\
        \hfill \break
        Lemma 1 says \\
        \circled{1} \alert{Leaf} states on the boundary path before the active point $s_j$ get a transition that expands an existing branch of the \alert{trie}. \\
        \circled{2} \alert{Non-leaf} states from the active point $s_j$ to end point $s_{j'}$ ($s_{j'}$ excluded) get a transition that initiates a new branch
    \end{frame}

    \begin{frame}[fragile,plain]\frametitle{Construction of Suffix Tree}
        Interpret in terms of suffix tree $STree(T^{i-1})$. \\
        \hfill \break
        Transitions from \circled{1} that expand an existing branch is implemented by updating the right pointer of each transition that represents the branch. \\
        \hfill \break
        Let $g'(s, (k, i - 1)) = r$ be such a transition. \\
        The right pointer has to point to the last position $i - 1$ of $T^{i-1}$. $\because r$ is a leaf $\Rightarrow$ a path leading to $r$ has to spell out a suffix of $T^{i-1}$ that does not occur elsewhere in $T^{i-1}$. \\
        $\therefore$ Updated transition is $g'(s, (k, i)) = r$. \\
        \hfill \break
        This only makes the string spelled out by the transition longer but does not change the states $s$ and $r$. Making all such updates would take too much time. We use a trick for this.
    \end{frame}

    \begin{frame}[fragile,plain]\frametitle{Construction of Suffix Tree}
        \begin{definitionblock}{Open transition}
            Any transition of $STree(T^{i-1})$ leading to a leaf
        \end{definitionblock}
        Open transitions are represented as $g'(s, (k, \infty)) = r$ \\
        Symbols $\infty$ can be replaced by $n = |T|$ after completing $STree(T)$ \\
        \hfill \break
        This way, transitions from \circled{1} is automatically done.
    \end{frame}

    \begin{frame}[fragile,plain]\frametitle{Construction of Suffix Tree}
        For transitions from \circled{2}, \\
        we need to find all $s_h,\  j \leq h < j'$, but $s_h$ might not be explicit. \\
        \hfill \break
        Let $h = j$ and let $(s, w)$ be the \underline{canonical reference pair} for $s_h$ (the active point). \\
        $s_h$ is on the boundary path of $STrie(T^{i-1})$ \\
        $\Rightarrow w $ is a suffix of $T^{i-1}$ \\
        $\Rightarrow (s, w) = (s, (k, i - 1))$ for some $k \leq i$ \\
        \hfill \break
    \end{frame}

    \begin{frame}[fragile,plain]\frametitle{Construction of Suffix Tree}
        We need to create a new branch starting from the state $(s, (k, i - 1))$. \\
        \hfill \break
        First, if $(s, (k, i - 1))$ is the end point, then done. \\
        Otherwise, $s_h = (s, (k, i - 1))$ has to be explicit in order to create a new branch from there. \\
        \hfill \break
        If $s_h$ is not explicit, create the explicit state $s_h$ by splitting the transition that contains the corresponding implicit state. \\
        After that, a $t_i$-transition from $s_h$ is created which is \\
        $g'(s_h, (i, \infty)) = s_{h'}$ where $s_{h'}$ is a new leaf. \\
        Moreover, suffix link $f'(s_h)$ is added if $s_h$ is created by splitting a transition.
    \end{frame}

    \begin{frame}[fragile,plain]\frametitle{Construction of Suffix Tree}
        Next the construction proceeds to $s_{h+1}$. \\
        \hfill \break
        Reference pair for $s_h = (s, (k, i - 1)) $ \\ 
        $\Rightarrow $ canonical reference pair for $s_{h+1} = canonize(f'(s), (k, i-1))$ where $canonize$ makes the pair canonical by updating the state and the left pointer. \\
        \hfill \break
        Repeat until $s_{j'}$ is found.
    \end{frame}

    \begin{frame}[fragile,plain]\frametitle{Construction of Suffix Tree}
        $update$ returns a reference pair for the end point $s_{j'}$ (only the state and the left pointer as right pointer is always $i - 1$) \\
        \hfill \break
        \LinesNumbered
        \begin{procedure}[H]
            \SetAlgoNoEnd
            \nonl $(s, (k, i - 1))$ the \underline{canonical reference pair} for the active point;
            \caption{update($s$, ($k$, $i$))}
            $oldr \leftarrow root$\;
            $(end\mhyphen point,r) \leftarrow test \mhyphen and \mhyphen split(s, (k, i - 1)),t_i)$\;
            \While {\Not $(end\mhyphen point)$}{
                create new transition $g'(r, (i, \infty)) = r'$ where $r'$ is a new state\;
                \lIf{$oldr \neq root$}{create new suffix link $f'(oldr) = r$}
                $oldr \leftarrow r$\;
                $(s, k) \leftarrow canonize(f'(s), (k, i-1))$\;
                $(end\mhyphen point, r) \leftarrow test\mhyphen and \mhyphen split(s,(k, i-1),t_i)$\;
            }
            \lIf{$oldr \neq root$}{create new suffix link $f'(oldr) = s$}
            \Return $(s, k)$.
        \end{procedure}
    \end{frame}

    \begin{frame}[fragile,plain]\frametitle{Construction of Suffix Tree}
        $test\mhyphen and \mhyphen split$ returns $($ `is end point?', explicit state$)$ \\
        \hfill \break
        \LinesNumbered
        \begin{procedure}[H]
            \SetAlgoNoEnd
            \nonl $(s, (k, p))$ is canonical
            \caption{test-and-split($s$, ($k$, $p$), $t$)}
            \uIf{$k \leq p$}{
                Let $g'(s, (k', p')) = s'$ be the $t_k$-transition from $s$\;
                \DontPrintSemicolon
                \lIf{$t = t_{k'+p-k+1}$} {
                    \Return (\True, $s$)
                } 
                \PrintSemicolon
                \uElse {
                    replace the $t_k$-transition above by transitions \hfill \break
                    $g'(s, (k', k'+p-k)) = r$ and $g'(r, (k'+p-k+1,p')) = s'$ where $r$ is a new state\;
                    \Return (\False, $r$)
                }
            }
            \uElse {
                \DontPrintSemicolon
                \lIf{there is no $t$-transition from $s$} {
                    \Return (\False, $s$)
                }
                \lElse {
                    \Return (\True, $s$).
                }
            }
        \end{procedure}
    \end{frame}

    \begin{frame}[fragile,plain]\frametitle{Construction of Suffix Tree}
        Procedure $test$-$and$-$split$ benefits from that $(s, (k, p))$ is canonical: The answer to the end point test can be found in constant time by considering only one transition from $s$
    \end{frame}

    \begin{frame}[fragile,plain]\frametitle{Construction of Suffix Tree}
        \LinesNumbered
        \begin{procedure}[H]
            \SetAlgoNoEnd
            \caption{canonize($s$, ($k$, $p$))}
            \DontPrintSemicolon
            \lIf{$p < k$}{
                \Return $(s, k)$
            }
            \PrintSemicolon
            \uElse {
                find the $t_k$-transition $g'(s, (k', p')) = s'$ from $s$\;
                \While{$p' - k' \leq p - k$}{
                    $k \leftarrow k + p' - k' + 1$\;
                    $s \leftarrow s'$\;
                    \lIf{$k \leq p$} {
                        find the $t_k$-transition $g'(s, (k', p')) = s'$ from $s$
                    }
                }
                \Return $(s, k)$.
            }
        \end{procedure}
        \hfill \break
        At return, $k$ can become $k', k' \geq k$.
    \end{frame}

    \begin{frame}[fragile,plain]\frametitle{Construction of Suffix Tree}
        To continue the construction for the next text symbol $t_{i+1}$, the active point of $STree(T^{i})$ has to be found
        \begin{factblock}{Fact 1}
            $s_j$ is the active point of $STree(T^{i-1})$ \\ $\Leftrightarrow s_j = \overline{t_j\cdots t_{i-1}}$ where $t_j\cdots t_{i-1}$ is the longest suffix of $T^{i-1}$ that occurs at least twice in $T^{i-1}$ \\
            \hfill \break
            $\because$ By the construction process, \\
            \begin{itemize}
                \item $t_j\cdots t_{i-1}$ has occured before $\Leftrightarrow s_j$ is not a leaf and \\
                \item $j$ is smallest meaning $t_j\cdots t_{i-1}$ is longest
            \end{itemize}
        \end{factblock}
    \end{frame}

    \begin{frame}[fragile,plain]\frametitle{Construction of Suffix Tree}
        \begin{factblock}{Fact 2}
            $s_{j'}$ is the end point of $STree(T^{i-1})$ \\ $\Leftrightarrow s_{j'} = \overline{t_{j'}\cdots t_{i-1}}$ where $t_{j'}\cdots t_{i-1}$ is the longest suffix of $T^{i-1}$ s.t. $t_{j'}\cdots \alert{t_{i-1}t_i}$ is a substring of $T^{i-1}$ \\
            \hfill \break
            $\because$ 
            \begin{itemize}
                \item by definition of end point and
                \item $j$ is smallest meaning $t_j\cdots t_{i-1}$ is longest
            \end{itemize}
        \end{factblock}
    \end{frame}

    \begin{frame}[fragile,plain]\frametitle{Construction of Suffix Tree}
        Combining the previous 2 facts gives \\
        \hfill \break
        $s_{j'}$ is the end point of $STree(T^{i-1})$ \\
        $\Rightarrow t_{j'}\cdots t_{i-1}\alert{t_i}$ is the longest suffix of $T^{\alert{i}}$ that occurs at least twice in $T^{\alert{i}}$ \\
        $\Leftrightarrow$ state $g'(s_{j'}, t_i)$ is the active point of $STree(T^{i})$ \\
        \hfill \break
        $\because t_{j'}\cdots t_{i-1}\alert{t_i}$ is a substring of $T^{\alert{i-1}}$ \\
        $\Rightarrow t_{j'}\cdots t_{i-1}t_i$ occurs at least twice in $T^{\alert{i}}$
    \end{frame}

    \begin{frame}[fragile,plain]\frametitle{Construction of Suffix Tree}
        \begin{theoremblock}{Lemma 2}
            $(s, (k, i - 1))$ is reference pair of the end point $s_{j'}$ of $STree(T^{i-1})$ \\
            $\Rightarrow (s, (k, i))$ is a reference pair of the active point of $STree(T^{i})$
        \end{theoremblock}
    \end{frame}

    \begin{frame}[fragile,plain]\frametitle{Construction of Suffix Tree}
        \LinesNumbered
        \begin{algorithm}[H]
            \SetAlgoNoEnd
            \caption{Construction of $STree(T)$ for string $T$ \\ = $t_1t_2\cdots \sharp$ in alphabet $\Sigma = \{t_{-1},\ldots,t_{-m}\}$; \\ $\sharp$ is the end marker not appearing elsewhere in $T$.}
            \label{alg:2}
            create states $root$ and $\bot$\;
            \lFor{$j \leftarrow 1,\ldots,m$} {
                create transition $g'(\bot, (-j, -j)) = root$
            }
            create suffix link $f'(root) = \bot$\;
            $s \leftarrow root$; $k \leftarrow 1$; $i \leftarrow 0$\;
            \While{$t_{i+1} \neq \sharp$}{
                $i \leftarrow i + 1$\;
                $(s, k) \leftarrow update(s, (k, i))$\;
                $(s, k) \leftarrow canonize(s, (k, i))$\;
            }
        \end{algorithm}
        \hfill \break
        Steps 7-8 are based on Lemma 2
    \end{frame}

    \begin{frame}[fragile,plain]\frametitle{Construction of Suffix Tree}
        \begin{theoremblock}{Theorem 2}
            Algorithm~\ref{alg:2} constructs the suffix tree $STree(T)$ for a string $T = t_1\cdots t_n$ on-line in time $\mathcal{O}(n)$
        \end{theoremblock}
    \end{frame}

    \begin{frame}[fragile,plain]\frametitle{Construction of Suffix Tree}
        Proof: \\
        \hfill \break
        The algorithm constructs $STree(T)$ through intermediate trees $STree(T^0)$, $STree(T^1)$,$\ldots$, $STree(T^n) = STree(T)$. \\
        It is on-line as to construct $STree(T^{i})$ it only needs access to the first $i$ symbols of $T$.
    \end{frame}

    \begin{frame}[fragile,plain]\frametitle{Construction of Suffix Tree}
        For the running time analysis, \\
        we divide the time requirement into two components, both turn out to be $\mathcal{O}(n)$. \\
        \hfill \break
        \diamonded{1} Total time for procedure $canonize$ \\
        \diamonded{2} The rest: The time for \\
        \hspace{24pt} repeatedly traversing the suffix link path from the present active point to the end point and\\
        \hspace{24pt} creating the new branches by update and\\
        \hspace{24pt} then finding the next active point by taking a transition from the end point\\
        \begin{definitionblock}{Visted States}
            Call the states (reference pairs) on these paths the \underline{visited states}
        \end{definitionblock}
    \end{frame}

    \begin{frame}[fragile,plain]\frametitle{Construction of Suffix Tree}
        \diamonded{2} takes time proportional to the total number of the visited states \\
        $\because$ the operations at each such state \\
        (create an explicit state and a new branch, \\
        follow an explicit or implicit suffix link, \\
        test for the end point) \\
        at each such state can be implemented in constant time as $canonize$ is excluded. \\
        \hfill \break
        (To be precise, this also requires that $|\Sigma|$ is bounded independently of $n$.)
    \end{frame}

    \begin{frame}[fragile,plain]\frametitle{Construction of Suffix Tree}
        Let $r_i$ be the active point of $STree(T^{i})$ for $0 \leq i \leq n$. \\
        The visited states between $r_{i-1}$ and $r_i$ are on a path that consists of some suffix links and one $t_i$-transition
        \begin{definitionblock}{Depth of a state}
            The \underline{depth} of a state is the length of the string spelled out on the transition path from $root$
        \end{definitionblock}
        Taking a suffix link decreases the depth of the current state by $1$.\\
        $\therefore$ The number of the visited states (including $r_{i-1}$ , excluding $r_i$) on the path is $depth(r_{i-1}) - depth(r_i) + 2$, \\
        \hfill \break
        and their total number is $\sum_{i=1}^{n} (depth(r_{i-1}) - depth(r_i) + 2) = depth(r_0) - depth(r_n) + 2n \leq 2n$. \\
        This implies \diamonded{2} takes time $\mathcal{O}(n)$
    \end{frame}

    \begin{frame}[fragile,plain]\frametitle{Construction of Suffix Tree}
        The time spent by each execution of $canonize$ is $\mathcal{O}(a + bq)$ \\
        where $a$ and $b$ are constants and \\
        $q$ is the number of executions of the body of the loop in steps 5-7 of $canonize$. \\
        $\therefore$ The total time spent by $canonize$ in all calls = \\
        $\mathcal{O}($number of the calls of $canonize$ $+$ \\
        the total number of the executions of the body of the loop$)$ \\
    \end{frame}

    \begin{frame}[fragile,plain]\frametitle{Construction of Suffix Tree}
        There are $\mathcal{O}(n)$ calls as there is one call for each visited state (either in step 6 of $update$ or directly in step 8 of Alg.~\ref{alg:2}.). \\
        Each execution of the body deletes a nonempty string from the left end of string $w = t_k\cdots t_p$ \\
        String $w$ can grow during the whole process only in step 6\footnote[1]{my correction} of Alg.\ref{alg:2} which catenates $t_i$ for $i = 1,\ldots,n$ to the right end of $w$. \\
        \hfill \break
        $\therefore$ a non-empty deletion is possible at most $n$ times. \\
        $\Rightarrow$ \diamonded{1} takes time $\mathcal{O}(n)$.

    \end{frame}



    \section{References}
    \begin{frame}[fragile,plain]\frametitle{References}
        Original Paper: \\
        \url{https://www.cs.helsinki.fi/u/ukkonen/SuffixT1withFigs.pdf} \\
        \hfill \break
        Wikipedia for the definition of Trie
    \end{frame}

    \end{document}
