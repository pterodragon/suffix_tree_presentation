\documentclass[compress,usenames,dvipsnames]{beamer}
\usepackage[T1]{fontenc}
\usepackage{beramono}
\usepackage{xcolor}
\usepackage[ruled]{algorithm2e}
\usepackage{tikz}
\usepackage{hyperref}
\usetheme{Warsaw}
\usecolortheme{crane}

\SetKw{Not}{not}
\mathchardef\mhyphen="2D

\newcommand*\circled[1]{\tikz[baseline=(char.base)]{
            \node[shape=circle,draw,inner sep=2pt] (char) {#1};}}

\let\oldnl\nl% Store \nl in \oldnl
\newcommand{\nonl}{\renewcommand{\nl}{\let\nl\oldnl}}% Remove line number for one line

\DeclareMathOperator{\atrans}{\mathnormal{-}}

\def\HiLi{\leavevmode\rlap{\hbox to \hsize{\color{yellow!50}\leaders\hrule height .8\baselineskip depth .5ex\hfill}}}
\usefonttheme[onlymath]{serif}

\newenvironment{definitionblock}[1]{
    \setbeamercolor{block title}{bg=cyan}
    \begin{block}{#1}}{\end{block}
}

\newenvironment{theoremblock}[1]{
    \setbeamercolor{block title}{bg=Emerald}
    \begin{block}{#1}}{\end{block}
}

\author{Wing}
\title{Suffix Tree (Ukkonen's algorithm)}  

\begin{document}
\date{\today} 

\frame[plain]{\titlepage} % [plain] means it doesn't show the section above the Header 

\frame[plain]{\frametitle{Table of contents}
    \small
    \tableofcontents[hideallsubsections]
}  

\section{Suffix Trie}
\begin{frame}[fragile,plain] \frametitle{Trie}
    \begin{block}{Trie}
        An ordered tree data structure used to store a dynamic set or map where the keys are usually strings
    \end{block}
\end{frame}

\begin{frame}[fragile,plain] \frametitle{Suffix Trie}
    \begin{figure}
        \begin{tikzpicture}[level distance=10mm,->,line width=1.5pt]
            \tikzstyle{every node}=[draw,circle,inner sep=1pt,minimum size=7.5pt,line width=0.5pt]
            \tikzstyle{sll}=[bend left,dashed,line width=0.5pt,->]
            \tikzstyle{slr}=[bend right,dashed,line width=0.5pt,->]
            \tikzstyle{level 1}=[sibling distance=10mm]
            \tikzstyle{level 2}=[sibling distance=20mm]
            \tikzstyle{level 3}=[sibling distance=10mm]
            \node [draw=none] (bottom) {$\bot$}
                child {node (root) {}
                    child {node {}
                        child{node {}
                            child{node {}
                                child{node {}
                                    child{node {}
                                        edge from parent
                                        node[auto,draw=none] {$o$}
                                    }
                                    child{node[draw=none] {}
                                        edge from parent[draw=none]
                                        node[auto,draw=none] {}
                                    }
                                    edge from parent
                                    node[auto,draw=none] {$a$}
                                }
                                child{node[draw=none] {}
                                    edge from parent[draw=none]
                                    node[auto,draw=none] {}
                                }
                                edge from parent
                                node[auto,draw=none] {$c$}
                            }
                            child{node {}
                                edge from parent
                                node[auto,draw=none] {$o$}
                            }
                            edge from parent
                            node[auto,draw=none] {$a$}
                        }
                        child{node[draw=none] {}
                            edge from parent[draw=none]
                            node[auto,draw=none] {}
                        }
                        edge from parent
                        node[auto,draw=none] {$c$}
                    }
                    child {node {}
                        child{node {}
                            child{node {}
                                child{node {}
                                    edge from parent
                                    node[auto,draw=none] {$o$}
                                }
                                child{node[draw=none] {}
                                    edge from parent[draw=none]
                                    node[auto,draw=none] {}
                                }
                                edge from parent
                                node[auto,draw=none] {$a$}
                            }
                            child{node[draw=none] {}
                                edge from parent[draw=none]
                                node[auto,draw=none] {}
                            }
                            edge from parent
                            node[auto,draw=none] {$c$}
                        }
                        child{node {}
                            edge from parent
                            node[auto,draw=none] {$o$}
                        }
                        edge from parent
                        node[auto,draw=none] {$a$}
                    }
                    child {node {}
                        edge from parent
                        node[auto,draw=none] {$o$}
                    }
                    edge from parent
                    node[draw=none,left] {$\Sigma$}
                }
                ;
                % \draw[style=slr] (root) to (bottom);
            \end{tikzpicture}
            \caption{Suffix Trie for $``cacao"$ (Suffix links omitted)} \label{fig:M1}
        \end{figure}
    \end{frame}

    \begin{frame}[fragile,plain] \frametitle{Suffix Trie}
        \begin{itemize}
            \item Proposed by Esko Ukkonen (University of Helsinki, Finland)
            \item An algorithm easier to grasp than the those in the literature at that time
            \item On-line algorithm: Processes the string symbol by symbol from left to right, and always has the suffix tree for the scanned
                part of the string ready
        \end{itemize}
    \end{frame}

    \begin{frame}[fragile,plain] \frametitle{Construction of Suffix Trie}
        \begin{definitionblock}{String}
            Let $T = t_1t_2\cdots t_n$ be a string over alphabet $\Sigma$
        \end{definitionblock}
        \begin{definitionblock}{Substring}
            Each string $x : T = uxv$ for some (possibly empty) string $u$ and $v$ is a \underline{substring} of $T$
        \end{definitionblock}
        \begin{definitionblock}{Suffix}
            $T_i = t_i\cdots t_n $ where $ 1 \leq i \leq n + 1 $
            \begin{itemize}
                \item $T_{n+1} = \epsilon$ is the \emph{empty} suffix
            \end{itemize}
        \end{definitionblock}
    \end{frame}

    \begin{frame}[fragile,plain] \frametitle{Construction of Suffix Trie}
        \begin{definitionblock}{Set of all suffixes of $T$}
            $\sigma(T)$
        \end{definitionblock}
        The suffix trie of $T$ is a trie representing $\sigma(T)$
    \end{frame}

    \begin{frame}[fragile,plain] \frametitle{Construction of Suffix Trie}
        \begin{definitionblock}{Suffix Trie}
            Denote suffix trie of $T$ as $STrie(T) = (Q \cup \{\bot\}, root, F, g, f)$
            \hfill \break
            \hfill \break
            Define such a trie as an augmented deterministic finite-state automaton which has a tree-shaped transition graph representing the trie for $\sigma(T)$
            \hfill \break
            \hfill \break
            augmented with
            \begin{itemize}
                \item $f$ : \underline{suffix function}
                \item $\bot$ : \underline{auxiliary state}
            \end{itemize}
        \end{definitionblock}
    \end{frame}

    \begin{frame}[fragile,plain] \frametitle{Construction of Suffix Trie}
        \begin{definitionblock}{Set $Q$ of the states of $STrie(T)$}
            The set $Q$ of the states of $STrie(T)$ can be put in a one-to-one correspondence with the substrings of $T$.
            \hfill \break
            \hfill \break
            Denote $\bar{x}$ the state that corresponds to a substring $x$
            \hfill \break
            Shorthand: $\bar{x} \sim x$
            \begin{itemize}
                \item $root \sim \epsilon$
                \item set $F$ of final states $\sim \sigma(T) $
            \end{itemize}

        \end{definitionblock}
    \end{frame}

    \begin{frame}[fragile,plain] \frametitle{Construction of Suffix Trie}
        \begin{definitionblock}{Transition function $g$}
            $
            \begin{cases}
                g(\bar{x}, a) = \bar{y} \  & \forall \bar{x}, \bar{y} \in Q : y = xa, \ \mbox{where} \ a \in \Sigma
                \\
                g(\bot, a) = root \ & \forall a \in \Sigma
            \end{cases}
            $
        \end{definitionblock}
        \begin{definitionblock}{Suffix function $f$}
            $\forall \bar{x} \in Q,$ \\
            $\begin{cases} f(\bar{x}) = \bar{y} & \mbox{if} \ \bar{x} \neq root, \mbox{then}\ x = ay, a \in \Sigma \\ f(root) = \bot \\
                f(\bot) \ \mbox{is undefined}
            \end{cases}$
            \end{definitionblock}
            $\bot \sim a^{-1} \ \forall a \in \Sigma$
            \hfill \break
            $a^{-1}a=\epsilon$
        \end{frame}

        \begin{frame}[fragile,plain] \frametitle{Construction of Suffix Trie}
            \begin{definitionblock}{Suffix Link}
                $f(r)$ is the suffix link of state $r$
            \end{definitionblock}

            \begin{definitionblock}{Prefix}
                $T^{i} = t_1\cdots t_i$ of $T$ for $0 \leq i \leq n $
            \end{definitionblock}
        \end{frame}

        \begin{frame}[fragile,plain] \frametitle{Construction of Suffix Trie}
            \begin{block}{Key observation}
                How is $STrie(T^i)$ obtained from $STrie(T^{i-1})$?
                \hfill \break
                \hfill \break
                The suffixes of $T^i$ can be obtained by catenating $t_i$ to the end of each suffix of $T^{i-1}$ and by adding an empty suffix, i.e.
                $$\sigma(T^i) = \sigma(T^{i-1})t_i \cup \{ \epsilon \} $$
            \end{block}
            $STrie(T^{i-1})$ accepts $\sigma(T^{i-1})$, to make it accept $\sigma(T^{i})$,
            examine $F_{i-1}$ of $STrie(T^{i-1})$
            \begin{itemize}
                \item $r \in F_{i-1}$ doesn't have a $t_i$-transition $\Rightarrow$ add transition $r \rightarrow$ new state
                \item $r \in F_{i-1}$ has a $t_i$-transition $\Rightarrow$ follow the transition to the old state
                \item All such states plus $root$ will be $F_i$ of $STrie(T^i)$
            \end{itemize}
        \end{frame}

        \begin{frame}[fragile,plain] \frametitle{Construction of Suffix Trie}
            How to find states $r \in F_{i-1}$ that get new transitions?
            \hfill \break
            \hfill \break
            From definition of the suffix function $f$,
            \hfill \break
            $r \in F_{i-1} \Leftrightarrow r = f^j (\overline{t_1\cdots t_{i-1}})$ for some $ 0 \leq j \leq i - 1$
            \begin{definitionblock}{Boundary path}
                Boundary path of $STrie(T^{i-1})$: \\
                Path starting from deepest state $\overline{t_1\cdots t_{i-1}}$ of $STrie(T^{i-1})$, following the suffix links and ending at $\bot$
            \end{definitionblock}
            $\therefore $ All states in $F_{i-1}$ are on the \underline{boundary path} of $STrie(T^{i-1})$
        \end{frame}

        \begin{frame}[fragile,plain] \frametitle{Construction of Suffix Trie}
            The boundary path is traversed. \\
            \hfill \break
            If a state $\bar{z}$ on the boundary path does not have a transition on $t_i$ yet,
            add a new state $\overline{zt_i}$ and a new transition $g(\bar{z}, t_i) = \overline{zt_i}$ \\
            \hfill \break
            To update $f$, new states $\overline{zt_i}$ are linked together with new suffix links starting from $\overline{t_1\cdots t_{i}}$. \\
            Obviously, this is the boundary path of $STrie(T^i)$
        \end{frame}

        \begin{frame}[fragile,plain] \frametitle{Construction of Suffix Trie}
            \begin{block}{Observation}
                The traversal over $F_{i-1}$ along the boundary path can be stopped immediately when the first state $\bar{z}$ is found s.t. state $\overline{zt_i}$ (and hence also transition $g(\bar{z}, t_i) = \overline{zt_i}$) already exists.
            \end{block}
            Let namely $\overline{zt_i}$ already be a state. \\
            \hfill \break
            Then $STrie(T^{i-1})$ has to contain state $\overline{z't_i}$ and transition $g(z', t_i) = \overline{z't_i} \ \forall z' = f^j(\bar{z}), j \leq 1$. \\
            In other words, if $\overline{zt_i}$ is a substring of $T_{i-1}$ then every suffix of $\overline{zt_i}$ is a substring of $T_{i-1}$.\\
            \hfill \break
            Such $\bar{z}$ must exist as $\bot$ is the last state on the boundary path that has the $t_i$-transition $\forall t_i$
        \end{frame}

        \begin{frame}[fragile,plain]\frametitle{Construction of Suffix Trie}
            $top$ denotes the state $\overline{t_1\cdots t_{i-1}}$
            \LinesNumbered
            \begin{algorithm}[H]
                \SetAlgoNoEnd
                $r \leftarrow top$ \\
                \While{$g(r, t_i)$ is undefined}{
                    create new state $r'$ and new transition $g(r, t_i) = r'$; \\
                    \lIf{$r \neq top$}{create new suffix link $f(oldr') = r'$}
                    $oldr' \leftarrow r'$; \\
                    $r \leftarrow f(r)$;
                }
                create new suffix link $f(oldr') = g(r, t_i)$ \\
                $top \leftarrow g(top, t_i)$.
                \caption{}
                \label{alg:1}
            \end{algorithm}
        \end{frame}

        \begin{frame}[fragile,plain]\frametitle{Construction of Suffix Trie}
            Running Algorithm~\ref{alg:1} for $t_i = t_1, t_2,\cdots, t_n$ visits each $\bar{x} \in Q$ once.
            \begin{theoremblock}{Theorem 1}
                Suffix trie $STrie(T)$ can be constructed in time proportional to
                the size of $STrie(T)$ which, in the worst case, is $\mathcal{O}(|T|^2)$.
            \end{theoremblock}
        \end{frame}

        \section{Suffix Tree}
        \begin{frame}[fragile,plain]\frametitle{Suffix Tree}
			Look at the figure.
            \begin{definitionblock}{Suffix tree}
                Suffix tree $STree(T)$ of $T$ is a data structure that represents $STrie(T)$ in space linear in the length $|T|$ of $T$
            \end{definitionblock}
        \end{frame}

        \begin{frame}[fragile,plain]\frametitle{Construction of Suffix Tree}
            \begin{definitionblock}{Explicit states}
                $Q' \cup \{\bot\}$ is the \underline{explicit states} of $STrie(T)$ \\
                \hfill \break
                $Q' \subseteq Q$ consists of all branching states and all leaves of $STrie(T)$ \\
                By definition, $root$ is included into the branching states
            \end{definitionblock}
            \begin{definitionblock}{Implicit states}
                Other states of $STrie(T)$ is the \underline{implicit states}
            \end{definitionblock}
        \end{frame}

        \begin{frame}[fragile,plain]\frametitle{Construction of Suffix Tree}
            \begin{definitionblock}{Generalized transition}
                $g'(s, w) = r$ in $STree(T)$ represents the string $w$ spelled out by the transition path in $STrie(T)$ between two explicit states $s$ and $r$ \\
                \hfill \break
                To save space, the string $w$ is represented as a pair $(k, p)$ of pointers: $t_k\cdots t_p = w$ \\
                Then $g'(s, (k, p)) = r$
            \end{definitionblock}
            Such pointers exist because there must be a suffix $T_i$ s.t. the transition path for $T_i$ in $STrie(T)$ goes through $s$ and $r$ \\
            \hfill \break
            {\color{red}Select the smallest such $i$, and let $k$ and $p$ point to the substring of this $T_i$ that is spelled out by the transition path from $s$ to $r$}
        \end{frame}

        \begin{frame}[fragile,plain]\frametitle{Construction of Suffix Tree}
            A transition $g'(s, (k, p)) = r$ is called an \underline{$a\atrans transition$} if $t_k = a$. Each $s$ can have at most one $a\atrans transition \ \forall a \in \Sigma$. \\
            \hfill \break
            Let $\Sigma = \{a_1, a_2, \ldots, a_m \}$. \\
            $g(\bot, a_j) = root$ is represented as $g(\bot, (-j, -j)) = root $ for $ j = 1, \ldots, m$.\\
            \hfill \break
            Hence suffix tree $STree(T)$ has two components: The tree itself and the string $T$.
        \end{frame}

        \begin{frame}[fragile,plain]\frametitle{Construction of Suffix Tree}
            $STree(T)$ is of linear size in $|T|$. \\
            $\because Q'$ has at most $|T|$ leaves ($\leq 1$ leaf for each nonempty suffix) \\
            $\Rightarrow Q'$ has to contain at most $|T| - 1$ branching states (when $|T| > 1$). \\
            \hfill \break
            $\therefore$ There can be at most $2|T| - 2$ transitions between the states in $Q'$ , each taking a constant space because of using pointers instead of an explicit string. \\
            {\color{red}$\Rightarrow$ In implementation, $g'$ can take $\mathcal{O}(|T|)$ space}
        \end{frame}

        \begin{frame}[fragile,plain]\frametitle{Construction of Suffix Tree}
            \begin{definitionblock}{Suffix function $f'$}
                Let $B \subset Q$ be the set of branching states in $STrie(T)$ \\
                \hfill \break
                $\forall \bar{x} \in B,$ \\
                $\begin{cases} f'(\bar{x}) = \bar{y} & \mbox{if} \ \bar{x} \neq root, \mbox{then}\ x = ay, a \in \Sigma, \bar{y} \in B \\
                    f'(root) = \bot
                \end{cases}$
                \end{definitionblock}
                $f'$ is well-defined $\because \bar{x} \in B \Rightarrow f'(\bar{x}) \in B$. \\
                These suffix links are explicitly represented. \underline{Implicit suffix links} are helpful but they are imaginary.
            \end{frame}

            \begin{frame}[fragile,plain]\frametitle{Construction of Suffix Tree}
                \begin{definitionblock}{Suffix Tree}
                    Denote suffix tree of $T$ as $STree(T) = (Q' \cup \{\bot\}, root, g', f')$
                \end{definitionblock}
            \end{frame}

            \begin{frame}[fragile,plain]\frametitle{Construction of Suffix Tree}
                \begin{definitionblock}{Reference pair}
                    $r = (s, w)$ \\
                    Refer to an {\color{red}explicit or implicit} state $r$ of a suffix tree by a \underline{reference pair} $(s, w)$ where \\
                    $s$ is some {\color{red}explicit} state that is an ancestor of $r$ and \\
                    $w$ is the string spelled out by the transitions from $s$ to $r$ in the corresponding suffix trie
                \end{definitionblock}
                \begin{definitionblock}{Canonical reference pair}
                    A reference pair is \underline{canonical} if $s$ is the closest ancestor of $r$ (and hence, $w$ is shortest possible)
                \end{definitionblock}
                If $r$ is explicit, canonical reference pair of $r = (r, \epsilon)$
            \end{frame}

            \begin{frame}[fragile,plain]\frametitle{Construction of Suffix Tree}
                Again, represent string $w$ as a pair $(k, p)$ of pointers s.t. $t_k \cdots t_p = w$. \\
                Then, reference pair $(s, w)$ gets form $(s, (k, p))$. $(s, \epsilon) = (s, (p + 1, p))$ \\
                \begin{alertblock}{Caution}
                    No contraints on $k$ and $p$ as long as $w$ spells the string
                \end{alertblock}
            \end{frame}

			\begin{frame}[fragile,plain]\frametitle{Construction of Suffix Tree}
				It is technically convenient to omit the final states in the definition of a suffix tree. \\
				When final states are necessary, either \\
				\begin{itemize}
					\item add a symbol $\#$ which doesn't occurs in $T$ at the end of $T$ or 
					\item traverse from leaf $\bar{T}$ to $root$ and make all the states on the path explicit
				\end{itemize}
                In many applications of $STree(T)$, the start location of each suffix is stored with the corresponding state. Such an augmented tree can be used as an index for finding any substring of $T$.
			\end{frame}

			\begin{frame}[fragile,plain]\frametitle{Construction of Suffix Tree}
                The algorithm for constructing $STree(T)$ will be patterned after Algorithm~\ref{alg:1}. \\
                Now, we make precise what Algorithm~\ref{alg:1} does. \\
                \hfill \break
                Let $s_1 = \overline{t_1\cdots t_{i-1}},s_2=\overline{t_2\cdots t_{i-1}},s_3,\ldots,s_i=root,s_{i+1}=\bot$ be the states of $STrie(T^{t-1})$ on the \underline{boundary path}. \\
                \hfill \break
                Let $j$ be the smallest index s.t. $s_j$ is not a leaf, and \\
				let $j'$ be the smallest index s.t. $s_{j'}$ has a $t_i\atrans transition$. \\
				\hfill \break
				As $s_1$ is a leaf and $\bot$ is a non-leaf that has a $t_i \atrans transition$, both $j$ and $j'$ are well-defined and $j \leq j'$
            \end{frame}

			\begin{frame}[fragile,plain]\frametitle{Construction of Suffix Tree}
                \begin{theoremblock}{Lemma 1}
                    Algorithm~\ref{alg:1} adds to $STrie(T^{i-1})$ a $t_i \atrans transition \ \forall s_h, 1 \leq h < j'$, s.t. \\
                    \begin{itemize}
                        \item for $1 \leq h < j$, \\
                            the new transition expands an old branch of the trie that ends at leaf $s_h$, \\
                        \item and for $j \leq h < j'$, the new transition initiates a new branch from $s_h$. \\
                    \end{itemize}
                    Algorithm~\ref{alg:1} does not create any other transitions
                \end{theoremblock}
            \end{frame}

            \begin{frame}[fragile,plain]\frametitle{Construction of Suffix Tree}
                \begin{definitionblock}{Active point}
                    $s_j$ is the \underline{active point} of ${\color{red}STrie}(T^{i-1})$
                \end{definitionblock}
                \begin{definitionblock}{End point}
                    $s_{j'}$ is the \underline{end point} of ${\color{red}STrie}(T^{i-1})$
                \end{definitionblock}
                These states are present, explicitly or implicitly, in $STree(T^{i-1})$ \\
                \hfill \break
                Lemma 1 says \\
                \circled{1} {\color{red}Leaf} states on the boundary path before the active point $s_j$ get a transition that expands an existing branch of the {\color{red}trie}. \\
                \circled{2} {\color{red}Non-leaf} states from the active point $s_j$ to end point $s_{j'}$ ($s_{j'}$ excluded) get a transition that initiates a new branch
            \end{frame}

			\begin{frame}[fragile,plain]\frametitle{Construction of Suffix Tree}
                Interpret in terms of suffix tree $STree(T^{i-1})$. \\
                \hfill \break
                Transitions from \circled{1} that expand an existing branch is implemented by updating the right pointer of each transition that represents the branch. \\
                \hfill \break
                Let $g'(s, (k, i - 1)) = r$ be such a transition. \\
                The right pointer has to point to the last position $i - 1$ of $T^{i-1}$. $\because r$ is a leaf $\Rightarrow$ a path leading to $r$ has to spell out a suffix of $T^{i-1}$ that does not occur elsewhere in $T^{i-1}$. \\
                $\therefore$ Updated transition is $g'(s, (k, i)) = r$. \\
                \hfill \break
                This only makes the string spelled out by the transition longer but does not change the states $s$ and $r$. Making all such updates would take too much time. We use a trick for this.
            \end{frame}

			\begin{frame}[fragile,plain]\frametitle{Construction of Suffix Tree}
                \begin{definitionblock}{Open transition}
                    Any transition of $STree(T^{i-1})$ leading to a leaf
                \end{definitionblock}
                Open transitions are represented as $g'(s, (k, \infty)) = r$ \\
                Symbols $\infty$ can be replaced by $n = |T|$ after completing $STree(T)$ \\
                \hfill \break
                This way, transitions from \circled{1} is automatically done.
            \end{frame}

			\begin{frame}[fragile,plain]\frametitle{Construction of Suffix Tree}
                For transitions from \circled{2}, \\
                we need to find all $s_h,\  j \leq h < j'$, but $s_h$ might not be explicit. \\
                \hfill \break
                Let $h = j$ and let $(s, w)$ be the \underline{canonical reference pair} for $s_h$ (the active point). \\
                $s_h$ is on the boundary path of $STrie(T^{i-1})$ \\
                $\Rightarrow w $ is a suffix of $T^{i-1}$ \\
                $\Rightarrow (s, w) = (s, (k, i - 1))$ for some $k \leq i$ \\
                \hfill \break
            \end{frame}

			\begin{frame}[fragile,plain]\frametitle{Construction of Suffix Tree}
                We need to create a new branch starting from the state $(s, (k, i - 1))$. \\
                \hfill \break
                First, if $(s, (k, i - 1))$ is the end point, then done. \\
                Otherwise, $s_h = (s, (k, i - 1))$ has to be explicit in order to create a new branch from there. \\
                \hfill \break
                If $s_h$ is not explicit, create the explicit state $s_h$ by splitting the transition that contains the corresponding implicit state. \\
                After that, a $t_i\atrans transition$ from $s_h$ is created which is \\
                $g'(s_h, (i, \infty)) = s_{h'}$ where $s_{h'}$ is a new leaf. \\
                Moreover, suffix link $f'(s_h)$ is added if $s_h$ is created by splitting a transition.
            \end{frame}

			\begin{frame}[fragile,plain]\frametitle{Construction of Suffix Tree}
                Next the construction proceeds to $s_{h+1}$. \\
                \hfill \break
                Reference pair for $s_h = (s, (k, i - 1)) $ \\ 
                $\Rightarrow $ canonical reference pair for $s_{h+1} = canonize(f'(s), (k, i-1))$ where $canonize$ makes the pair canonical by updating the state and the left pointer. \\
                \hfill \break
                Repeat until $s_{j'}$ is found.
            \end{frame}

            \begin{frame}[fragile,plain]\frametitle{Construction of Suffix Tree}
                \LinesNumbered
                \begin{procedure}[H]
                    \SetAlgoNoEnd
                    \nonl $(s, (k, i - 1))$ the \underline{canonical reference pair} for the active point;
                    \caption{update($s$, ($k$, $i$))}
                    $oldr \leftarrow root$;
                    $(end\mhyphen point,r) \leftarrow test \mhyphen and \mhyphen split(s, (k, i - 1)),t_i)$; \\
                    \While {\Not $(end\mhyphen point)$}{
                        create new transition $g'(r, (i, \infty)) = r'$ where $r'$ is a new state; \\
                        \lIf{$oldr \neq root$}{create new suffix link $f'(oldr) = r'$}
                        $oldr \leftarrow r$; \\
                        $(s, k) \leftarrow canonize(f'(s), (k, i-1))$; \\
                        $(end\mhyphen point, r) \leftarrow test\mhyphen and \mhyphen split(s,(k, i-1),t_i)$;
                    }
                    \lIf{$oldr \neq root$}{create new suffix link $f'(oldr) = s'$}
                    \Return $(s, k)$.
                \end{procedure}
            \end{frame}

            %         \frame[plain]{\frametitle{test}
            %             $$\sum_{n=1}^{\infty} 2^{-n} = 1$$ \\
            %             $$\mathcal{O}(n)$$
            %             \begin{block}{block name}
            %                 block
            %             \end{block}
            %             \pause
            %             \begin{exampleblock}{exampleblock name}
            %                 exampleblock
            %             \end{exampleblock}
            %             \begin{alertblock}{alertblock name}
            %                 alertblock
            %             \end{alertblock}
            %         }
            %         \section{Usage of Suffix Tree}
            %         \begin{frame}[fragile,plain]
            %             hi
            %         \end{frame}

            \section{References}
            \begin{frame}[fragile,plain]\frametitle{References}
                Original Paper: \\
                \url{https://www.cs.helsinki.fi/u/ukkonen/SuffixT1withFigs.pdf} \\
                \hfill \break
                Wikipedia for the definition of Trie
            \end{frame}

            \end{document}
